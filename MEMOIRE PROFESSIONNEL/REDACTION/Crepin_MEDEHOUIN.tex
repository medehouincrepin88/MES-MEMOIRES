%\title{University of Bristol Thesis Template}
\RequirePackage[l2tabu]{nag}		% Warns for incorrect (obsolete) LaTeX usage
%
%

% Memoir class loads useful packages by default (see manual).
%\documentclass[a4paper,11pt,leqno,openbib,oldfontcommands, openany]{memoir} %add 'draft' to turn draft option on (see below)
\documentclass[a4paper,12pt,leqno,oldfontcommands,openany]{memoir}  % Retirez "openbib" 
%
%
\usepackage[table, xcdraw, dvipsnames]{xcolor}
% Adding metadata:
\usepackage{datetime}
\usepackage{ifpdf}
\ifpdf
\pdfinfo{
	/Author (Author's name)
	/Title (MSc Thesis)
	/Keywords (One; Two;Three)
	/CreationDate (D:\pdfdate)
}
\fi
% When draft option is on. 
\ifdraftdoc 
\usepackage{draftwatermark}				%Sets watermarks up.
\SetWatermarkScale{0.3}
\SetWatermarkText{\bf Draft: \today}
\fi
%
% Declare figure/table as a subfloat.
\newsubfloat{figure}
\newsubfloat{table}
% Better page layout for A4 paper, see memoir manual.
\settrimmedsize{297mm}{210mm}{*}
\setlength{\trimtop}{0pt} 
\setlength{\trimedge}{\stockwidth} 
\addtolength{\trimedge}{-\paperwidth} 
\settypeblocksize{634pt}{448.13pt}{*} 
\setulmargins{4cm}{*}{*} 
\setlrmargins{*}{*}{1.5} 
\setmarginnotes{17pt}{51pt}{\onelineskip} 
\setheadfoot{\onelineskip}{2\onelineskip} 
\setheaderspaces{*}{2\onelineskip}{*} 
\checkandfixthelayout
%
\frenchspacing
% Font with math support: New Century Schoolbook
%\usepackage{fouriernc}
\usepackage[T1]{fontenc}
%
% UoB guidelines:
%
% Text should be in double or 1.5 line spacing, and font size should be
% chosen to ensure clarity and legibility for the main text and for any
% quotations and footnotes. Margins should allow for eventual hard binding.
%
% Note: This is automatically set by memoir class. Nevertheless \OnehalfSpacing 
% enables double spacing but leaves single spaced for captions for instance. 
\OnehalfSpacing 
%
% Sets numbering division level
\setsecnumdepth{subsection} 
\maxsecnumdepth{subsubsection}
%
% Chapter style (taken and slightly modified from Lars Madsen Memoir Chapter 
% Styles document
\usepackage{calc,soul,mathpazo}
\makeatletter 
\newlength\dlf@normtxtw 
\setlength\dlf@normtxtw{\textwidth} 
\newsavebox{\feline@chapter} 
\newcommand\feline@chapter@marker[1][4cm]{%
	\sbox\feline@chapter{% 
		\resizebox{!}{#1}{\fboxsep=1pt%
			\colorbox{MidnightBlue}{\color{white}\thechapter}% 
	}}%
	\rotatebox{90}{% 
		\resizebox{%
			\heightof{\usebox{\feline@chapter}}+\depthof{\usebox{\feline@chapter}}}% 
		{!}{\scshape\so\@chapapp}}\quad%
	\raisebox{\depthof{\usebox{\feline@chapter}}}{\usebox{\feline@chapter}}%
} 
\newcommand\feline@chm[1][4cm]{%
	\sbox\feline@chapter{\feline@chapter@marker[#1]}% 
	\makebox[0pt][c]{% aka \rlap
		\makebox[1cm][r]{\usebox\feline@chapter}%
}}
\makechapterstyle{daleifmodif}{
	\renewcommand\chapnamefont{\normalfont\Large\scshape\raggedleft\so} 
	\renewcommand\chaptitlefont{\normalfont\Large\bfseries\scshape} 
	\renewcommand\chapternamenum{} \renewcommand\printchaptername{} 
	\renewcommand\printchapternum{\null\hfill\feline@chm[2.5cm]\par} 
	\renewcommand\afterchapternum{\par\vskip\midchapskip} 
	\renewcommand\printchaptertitle[1]{\color{MidnightBlue}\chaptitlefont\raggedleft ##1\par}
} 
\makeatother 
\chapterstyle{daleifmodif}
%
% UoB guidelines:
%
% The pages should be numbered consecutively at the bottom centre of the
% page.
\makepagestyle{myvf} 
\makeoddfoot{myvf}{}{\thepage}{} 
\makeevenfoot{myvf}{}{\thepage}{} 
\makeheadrule{myvf}{\textwidth}{\normalrulethickness} 
\makeevenhead{myvf}{\small\textsc{\leftmark}}{}{} 
\makeoddhead{myvf}{}{}{\small\textsc{\rightmark}}
\pagestyle{myvf}
%
% Oscar's command (it works):
% Fills blank pages until next odd-numbered page. Used to emulate single-sided
% frontmatter. This will work for title, abstract and declaration. Though the
% contents sections will each start on an odd-numbered page they will
% spill over onto the even-numbered pages if extending beyond one page
% (hopefully, this is ok).
\newcommand{\clearemptydoublepage}{\newpage{\thispagestyle{empty}\cleardoublepage}}
%
%
% Creates indexes for Table of Contents, List of Figures, List of Tables and Index
\makeindex
%
% The import command enables each chapter tex file to use relative paths when
% accessing supplementary files. For example, to include
% chapters/brewing/images/figure1.png from chapters/brewing/brewing.tex we can
% use
% \includegraphics{images/figure1}
% instead of
% \includegraphics{chapters/brewing/images/figure1}
\usepackage{import}

% Add other packages needed for chapters here. For example:
\usepackage{lipsum}					%Needed to create dummy text
\usepackage{amsfonts} 					%Calls Amer. Math. Soc. (AMS) fonts
\usepackage[centertags]{amsmath}			%Writes maths centred down
\usepackage{stmaryrd}					%New AMS symbols
\usepackage{amssymb}					%Calls AMS symbols
\usepackage{amsthm}					%Calls AMS theorem environment
\usepackage{newlfont}					%Helpful package for fonts and symbols
\usepackage{layouts}					%Layout diagrams
\usepackage{graphicx}					%Calls figure environment
\usepackage{longtable,rotating}			%Long tab environments including rotation. 
\usepackage[utf8]{inputenc}			%Needed to encode non-english characters 
%directly for mac
\usepackage{colortbl}					%Makes coloured tables
\usepackage{wasysym}					%More math symbols
\usepackage{mathrsfs}					%Even more math symbols
\usepackage{float}						%Helps to place figures, tables, etc. 
\usepackage{verbatim}					%Permits pre-formated text insertion
\usepackage{upgreek }					%Calls other kind of greek alphabet
\usepackage{latexsym}					%Extra symbols



\usepackage[style=apa,sortcites=true]{biblatex}
\addbibresource{bibliography.bib} 

\usepackage{fontawesome5} % Package essentiel pour les icônes
% Définition de couleurs (optionnel)
\definecolor{niceblue}{RGB}{70, 130, 180}
\definecolor{nicered}{RGB}{220, 80, 70}


\usepackage{enumitem}     % Package pour personnaliser les listes





\usepackage{url}						%Supports url commands
% \usepackage{etex}						%eTeXÕs extended support for counters
% \usepackage{fixltx2e}					%Eliminates some in felicities of the 
\usepackage[french]{babel}		%For languages characters and hyphenation
\usepackage{color}                    				%Creates coloured text and background
\usepackage[colorlinks=true,
allcolors=MidnightBlue]{hyperref}              %Creates hyperlinks in cross references
\usepackage{memhfixc}					%Must be used on memoir document 
%class after hyperref
\usepackage{enumerate}					%For enumeration counter
\usepackage{footnote}					%For footnotes
\usepackage{microtype}					%Makes pdf look better.
\usepackage{rotfloat}					%For rotating and float environments as tables, 
%figures, etc. 
\usepackage{alltt}						%LaTeX commands are not disabled in 
%verbatim-like environment
\usepackage[version=0.96]{pgf}			%PGF/TikZ is a tandem of languages for producing vector graphics from a 
\usepackage{tikz}						%geometric/algebraic description.
\usetikzlibrary{arrows,shapes,decorations,
	automata,backgrounds,
	petri,topaths}				%To use diverse features from tikz	



% \usepackage{etex}						%eTeXÕs extended support for counters
% \usepackage{fixltx2e}					%Eliminates some in felicities of the 
%original LaTeX kernel
%For languages characters and hyphenation

\PassOptionsToPackage{table, xcdraw, dvipsnames}{xcolor}
\usepackage{xcolor}


\usepackage[autostyle=true]{csquotes} % Required to generate language-dependent quotes in the bibliography

\AtBeginEnvironment{enquote}{\itshape} % change the quote words to italics style

\usepackage{rotating} % Required to add rotated big tables

\usepackage[normalem]{ulem} % Required to add dashed underlines


\usepackage{rotating} % Required to add rotated big tables


\usepackage{pifont}	       



\usepackage{pdfpages}		       






% \printglossaries below creates a list of abbreviations. 
% \gls, \newacronym, \acrlongr a related commands are then 
% used throughout the text, so that latex can automatically
% keep track of which abbreviations have already been defined in the text.
\usepackage[acronym,toc]{glossaries}
\makeglossaries
%							
%Reduce widows  (the last line of a paragraph at the start of a page) and orphans 
% (the first line of paragraph at the end of a page)
\widowpenalty=1000
\clubpenalty=1000
%
% New command definitions for my thesis
%
\newcommand{\keywords}[1]{\par\noindent{\small{\bf Keywords:} #1}} %Defines keywords small section
\newcommand{\parcial}[2]{\frac{\partial#1}{\partial#2}}                             %Defines a partial operator
\newcommand{\vectorr}[1]{\mathbf{#1}}                                                        %Defines a bold vector
\newcommand{\vecol}[2]{\left(                                                                         %Defines a column vector
	\begin{array}{c} 
		\displaystyle#1 \\
		\displaystyle#2
	\end{array}\right)}
\newcommand{\mados}[4]{\left(                                                                       %Defines a 2x2 matrix
	\begin{array}{cc}
		\displaystyle#1 &\displaystyle #2 \\
		\displaystyle#3 & \displaystyle#4
	\end{array}\right)}
\newcommand{\pgftextcircled}[1]{                                                                    %Defines encircled text
	\setbox0=\hbox{#1}%
	\dimen0\wd0%
	\divide\dimen0 by 2%
	\begin{tikzpicture}[baseline=(a.base)]%
		\useasboundingbox (-\the\dimen0,0pt) rectangle (\the\dimen0,1pt);
		\node[circle,draw,outer sep=0pt,inner sep=0.1ex] (a) {#1};
	\end{tikzpicture}
}
\newcommand{\range}[1]{\textnormal{range }#1}                                             %Defines range operator
\newcommand{\innerp}[2]{\left\langle#1,#2\right\rangle}                                 %Defines inner product
\newcommand{\prom}[1]{\left\langle#1\right\rangle}                                         %Defines average operator
\newcommand{\tra}[1]{\textnormal{tra} \: #1}                                                       %Defines trace operator
\newcommand{\sign}[1]{\textnormal{sign\,}#1}                                                   %Defines sign operator
\newcommand{\sech}[1]{\textnormal{sech} #1}                                                  %Defines sech
\newcommand{\diag}[1]{\textnormal{diag} #1}                                                    %Defines diag operator
\newcommand{\arcsech}[1]{\textnormal{arcsech} #1}                                       %Defines arcsech
\newcommand{\arctanh}[1]{\textnormal{arctanh} #1}                                         %Defines arctanh
%Change tombstone symbol
\newcommand{\blackged}{\hfill$\blacksquare$}
\newcommand{\whiteged}{\hfill$\square$}
\newcounter{proofcount}
\renewenvironment{proof}[1][\proofname.]{\par
	\ifnum \theproofcount>0 \pushQED{\whiteged} \else \pushQED{\blackged} \fi%
	\refstepcounter{proofcount}
	\normalfont 
	\trivlist
	\item[\hskip\labelsep
	\itshape
	{\bf\em #1}]\ignorespaces
}{%
	\addtocounter{proofcount}{-1}
	\popQED\endtrivlist
}
%
%
% New definition of square root:
% it renames \sqrt as \oldsqrt
\let\oldsqrt\sqrt
% it defines the new \sqrt in terms of the old one
\def\sqrt{\mathpalette\DHLhksqrt}
\def\DHLhksqrt#1#2{%
	\setbox0=\hbox{$#1\oldsqrt{#2\,}$}\dimen0=\ht0
	\advance\dimen0-0.2\ht0
	\setbox2=\hbox{\vrule height\ht0 depth -\dimen0}%
	{\box0\lower0.4pt\box2}}
%
% My caption style
\newcommand{\mycaption}[2][\@empty]{
	\captionnamefont{\scshape} 
	\changecaptionwidth
	\captionwidth{0.9\linewidth}
	\captiondelim{.\:} 
	\indentcaption{0.75cm}
	\captionstyle[\centering]{}
	\setlength{\belowcaptionskip}{10pt}
	\ifx \@empty#1 \caption{#2}\else \caption[#1]{#2}
}
%
% My subcaption style
\newcommand{\mysubcaption}[2][\@empty]{
	\subcaptionsize{\small}
	\hangsubcaption
	\subcaptionlabelfont{\rmfamily}
	\sidecapstyle{\raggedright}
	\setlength{\belowcaptionskip}{10pt}
	\ifx \@empty#1 \subcaption{#2}\else \subcaption[#1]{#2}
}
%
%An initial of the very first character of the content
\usepackage{lettrine}
\newcommand{\initial}[1]{%
	\lettrine[lines=3,lhang=0.33,nindent=0em]{
		\color{gray}
		{\textsc{#1}}}{}}
%
% Theorem styles used in my thesis
%
\theoremstyle{plain}
\newtheorem{theo}{Theorem}[chapter]
\theoremstyle{plain}
\newtheorem{prop}{Proposition}[chapter]
\theoremstyle{plain}
\theoremstyle{definition}
\newtheorem{dfn}{Definition}[chapter]
\theoremstyle{plain}
\newtheorem{lema}{Lemma}[chapter]
\theoremstyle{plain}
\newtheorem{cor}{Corollary}[chapter]
\theoremstyle{plain}
\newtheorem{resu}{Result}[chapter]
%
% Hyphenation for some words
%
\hyphenation{res-pec-tively}
\hyphenation{mono-ti-ca-lly}
\hyphenation{hypo-the-sis}
\hyphenation{para-me-ters}
\hyphenation{sol-va-bi-li-ty}
%
%

\newacronym{gcm}{GCM}{Global Climate Model}
\newacronym{rcm}{RCM}{Regional Climate Model}
\newacronym{cpm}{CPM}{Convection-Permitting Model}


\usepackage{ulem} % dans le préambule
\usepackage{calligra}


\usepackage{array}

\usepackage{multirow}


\usepackage{tabularray}

\usepackage[object=vectorian]{pgfornament}


\usepackage{geometry}

\geometry{
	paper=a4paper, % Change to letterpaper for US letter
	left=2.5cm, % Left margin
	right=2.5cm, % Right margin
	bindingoffset=.5cm, % Binding offset
	top=1.5cm, % Top margin
	bottom=1.5cm, % Bottom margin
	%showframe, % Uncomment to show how the type block is set on the page
}



% Réinitialise les compteurs au début du document
\AtBeginDocument{%
	\setcounter{figure}{0}
	\setcounter{table}{0}
}

%\renewcommand{\thechapter}{\Roman{chapter}}
\renewcommand{\thechapter}{\arabic{chapter}}

%\renewcommand{\thesection}{\arabic{section}}

\usepackage{mdframed}
\usepackage{adjustbox} % Pour l'alignement vertical



\usepackage{amsmath}
\usepackage{amssymb}
%\usepackage[framemethod=default]{mdframed} % Version simplifiée pour éviter les conflits
%\usepackage{xcolor}

% Style de cadre simplifié et compatible
\mdfdefinestyle{mybox}{
	linewidth=1pt,
	linecolor=black,
	backgroundcolor=gray!10,
	innertopmargin=10pt,
	innerbottommargin=10pt,
	innerleftmargin=15pt, % Augmenté pour mieux contenir l'équation
	innerrightmargin=15pt,
	skipabove=10pt,
	skipbelow=10pt
}


\begin{document}
	% UoB guidlines:

\renewcommand{\tablename}{Tableau}


	\frontmatter
	\pagenumbering{roman}
	%
	\begin{titlingpage}
	% Pour le cadre 
	\color{black}
	\begin{tikzpicture}[remember picture,overlay]
		\draw[very thick]
		([yshift=-25pt,xshift=25pt]current page.north west)--
		([yshift=-25pt,xshift=-25pt]current page.north east)--
		([yshift=25pt,xshift=-25pt]current page.south east)--
		([yshift=25pt,xshift=25pt]current page.south west)--cycle;
	\end{tikzpicture}
	\begin{tikzpicture}[remember picture,overlay]
		\draw[very thick]
		([yshift=-20pt,xshift=20pt]current page.north west)--
		([yshift=-20pt,xshift=-20pt]current page.north east)--
		([yshift=20pt,xshift=-20pt]current page.south east)--
		([yshift=21pt,xshift=21pt]current page.south west)--cycle;
	\end{tikzpicture}
	\color{black}
	
	\begin{SingleSpace}
		\begin{center}
			RÉPUBLIQUE DE CÔTE D'IVOIRE \\
			\includegraphics[scale = 0.19]{logos/CI.png}\\ % Premier logo (inchangé)
			\textit{Union - Discipline - Travail}\\
			\vspace{0.3cm}
			- - - - - - - - -\\
		\begin{center}
			\begin{minipage}{0.42\textwidth}
				\centering
		Ministère de l’Économie, du Plan et du Développement \\
		\vspace{0.3cm}
		\includegraphics[scale=0.1]{logos/ENSEA.jpg} \\
		\vspace{0.4cm}
		École Nationale Supérieure de Statistique et d’Économie Appliquée 
			\end{minipage}
			\hfill
			\begin{minipage}{0.5\textwidth}
		\centering
		Ministère de la santé, de l’hygiène publique et de la couverture maladie universelle\\
			\vspace{0.5cm}
		\includegraphics[scale=0.35]{logos/pacci.jpeg} \\
		Programme ANRS Coopération \par Côte d’Ivoire %/ Centre de recherche avec statut d'Organisation Non Gouvernementale
		
		
			\end{minipage}
		\end{center}
		
		
		

		
		
		
		\vspace{0.1cm}
		
		
		
		\end{center}
		
	%	\calccentering{\unitlength} 
		
		\begin{center}
			\begin{center}
				\begin{tikzpicture}[overlay, remember picture]
					\node (A) at (-8.5,0) {};
					\node (B) at (8,0) {};
					\draw [black, scale=2] (A) to [ornament=88] (B);
				\end{tikzpicture}
			\end{center}
			
			\vspace{0.5cm}
			
			{\bf \normalsize \sc  FACTEURS ASSOCIES A LA PRISE DE POIDS CHEZ LES PATIENTS SOUS BITHERAPIE DANS L'ESSAI MODERATO}
			
			\begin{center}
				\begin{tikzpicture}[overlay, remember picture]
					\node (A) at (-8.5,0) {};
					\node (B) at (8,0) {};
					\draw [black, scale=2] (A) to [ornament=88] (B);
				\end{tikzpicture}
			\end{center}
			
\vspace{0.2cm}			
	\begin{center}
		% Minipage gauche : auteur
		\begin{minipage}[t]{0.4\textwidth}
			Auteur : \par 
			\vspace{0.2cm} 
			Crépin MEDEHOUIN  \par 
			\textit{\small Élève ingénieur statisticien économiste}
		\end{minipage}%
		\hfill
		% Minipage droite : encadrant aligné à droite
		\begin{minipage}[t]{0.45\textwidth}
			\raggedleft % ceci aligne tout à droite dans le minipage
			Encadrant :  \par 
			\vspace{0.2cm}  
			Dr. Fatoumata FADIGA  \par 
			\textit{\small Chef de projet International de MODERATO}
		\end{minipage}
	\end{center}
	
	
	
		
		\begin{flushright}
	\begin{minipage}{0.9\textwidth}

\vspace{0.7cm}
\begin{mdframed}[
	linewidth=1.3pt,
	leftmargin=50pt,
	rightline=false,
	topline=false,
	bottomline=false,
	innerleftmargin=10pt,
	innerrightmargin=10pt,
	innertopmargin=0pt,
	innerbottommargin=0pt,
	skipabove=0pt,
	skipbelow=0pt
	]
	
	\textbf{Composition du jury}
	
	\vspace{0.7em}
	{\footnotesize  \textbf{Dr Rosine  MOSSO},
	  \textit{\footnotesize Enseignant-Chercheur, ENSEA - Abidjan} 	\vspace{0.1cm}
	
	\textbf{Dr Raïmi  FASSASSI},
	 \textit{\footnotesize Enseignant-Chercheur, ENSEA - Abidjan} \vspace{0.1cm}
	
	\textbf{Dr Boris BAFFO},
	 \textit{\footnotesize Enseignant-Chercheur, ENSEA - Abidjan}}
\end{mdframed}
	\end{minipage}
	\hfill
	\begin{minipage}{0.01\textwidth}
	\end{minipage}
\end{flushright}

	\vspace{0.4cm}


\textit{{\small Mémoire professionnel soumis à l'ENSEA d'Abidjan pour l’obtention du diplôme d’Ingénieur Statisticien Économiste }} 	\vspace{0.5cm}
			
 Septembre 2025
			
			
			
			
			
			
			
			
			
			
			
			
%			\vspace{0.5cm}
%			Auteur : \par \vspace{0.2cm} Crépin MEDEHOUIN  \par \textit{\scriptsize Élève ingénieur statisticien économiste}\\
%			
%			\vspace{1cm}
%			Encadrant : \par \vspace{0.2cm} Dr. Fatou FADIGA  \par \textit{ \scriptsize ....}\\
%			
%			\vspace{1cm}
%			\textit{Mémoire professionnelle soumis à l'École Nationale Supérieure de Statistique et d'Économie Appliquée}
			
%			\vspace{1.3cm}
%			\text{Abidjan, Côte d'Ivoire} \par Septembre 2025
			
		\end{center}
	\end{SingleSpace}
\end{titlingpage}

	%
	\pagestyle{empty}

\addtocounter{page}{-1}

\vspace*{0.5cm}
\begin{center}
\begin{center}
	\begin{tikzpicture}[overlay, remember picture]
		\node (A) at (-9,0) {};
		\node (B) at (9,0) {};
		\draw [black, scale=2] (A) to [ornament=88] (B);
	\end{tikzpicture}
\end{center}
\vspace*{2cm}

{\bf \huge   Facteurs associés à la prise de poids chez les patients sous bithérapie dans l'essai MODERATO}
\end{center}

  \par
\vspace{2cm}

\begin{center}
	Par
	
	\vspace{10mm}
	{\bf Crépin MEDEHOUIN } \\

	\textit{Élève ingénieur statisticien économiste\\ en dernière année de formation}
\end{center}

	\vspace{1cm}

\begin{center}
	Pour
	
	\vspace{1cm}
	{\bf l'obtention du diplôme} \\
	\textit{d'Ingénieur Statisticien Économiste}
\end{center}




\vspace{5cm}
\begin{center}
	Mémoire professionnel \\
	\vspace{0.3cm}
	au sein du\\
	\vspace{0.3cm}
	Programme ANRS Coopération Côte d’Ivoire\\
	
\end{center}



	%
	

\chapter*{DÉDICACES}
\addcontentsline{toc}{chapter}{DÉDICACES}



%\chapter*{{Dédicaces}\addcontentsline{toc}{chapter}{Dédicaces}}

\begin{SingleSpace}

\centering
%\emph{A Mes frères, Serge, Michael, Joël, B. Stéphane, A. Christian, P. Kevin\\ et \\ mes petites soeurs Fortune et Eunice.}


\vfill
\begin{center}
	{\calligra \Huge  À toute ma famille}
\end{center}


\vfill

\end{SingleSpace}
	%
	

\pagestyle{plain}





\chapter*{DÉCHARGE}
\addcontentsline{toc}{chapter}{DÉCHARGE}


%\chapter*{{Décharge}\addcontentsline{toc}{chapter}{Décharge}}
\begin{SingleSpace}
\begin{quote}


\initial{L}es affirmations, interprétations et conclusions exprimées dans le présent document sont celles de l'auteur et ne reflètent pas nécessairement les vues de l’École Nationale Supérieure de Statistique et d’Économie Appliquée (ENSEA) d’Abidjan  ni celles du Programme ANRS Coopération Côte d’Ivoire.



\vspace{2cm} 
\begin{flushright}
	\hfill Signé: \underline{\hspace{0.3cm} C.M.  \hspace{0.2cm}}\\[2em]  % This prints a line for the signature
	\hfill Date: \underline{\hspace{0.3cm} \usdate\today \hspace{0.3cm}}\\ % This prints a line to write the date
\end{flushright}



\end{quote}
\end{SingleSpace}
\clearpage
	%
	%
% file: dedication.tex
% author: V?ctor Bre?a-Medina
% description: Contains the text for thesis dedication
%


\chapter*{REMERCIEMENTS}
\addcontentsline{toc}{chapter}{REMERCIEMENTS}



%\chapter*{{Remerciements}\addcontentsline{toc}{chapter}{Remerciements}}
\begin{SingleSpace}

\initial{A}u terme de ce travail, nous tenons à manifester nos sincères gratitudes à tous ceux qui, de près ou de loin, ont contribué à la réussite de notre formation et à la rédaction de ce mémoire. \\



Nous adressons particulièrement nos remerciements : 

\begin{itemize}[leftmargin=*, noitemsep]
	\item[\faUserTie{\color{niceblue}}] au Dr Raoul MOH, Directeur exécutif du Programme PAC-CI et Coordinateur MODERATO, pour nous avoir permis de réaliser ce stage au sein de sa structure et pour son encadrement particulier ; \vspace{0.1cm}	
	
	\item[\faUser] au Dr Fatoumata FADIGA, mon encadrante (Chef de projet International), pour sa disponibilité et ses conseils précieux tout au long de ce travail ; \vspace{0.1cm}	
	
	\item[\faUserTie] au Dr Serge NIANGORAN, biostatisticien à PAC-CI, pour son aide précieuse dans la modélisation du thème traité.
\end{itemize}

\vspace{0.3cm}	




Par ailleurs, nous tenons à remercier tout le corps professoral et administratif de l’ENSEA, et en particulier : 

\begin{itemize}[leftmargin=*, noitemsep]
	\item[\faUserTie] le Dr Hugues KOUADIO, Directeur Général de l'ENSEA, pour le cadre de formation offert ; 
	
	\item[\faUserTie] le Dr Romaric COULIBALY, Directeur des Études de la filière ISE, pour sa discipline et son engagement, qui ont contribué au bon déroulement de la formation.
\end{itemize}


\vspace{0.3cm}	

Nos remerciements vont également à : 

\begin{itemize}[leftmargin=*, noitemsep]
	\item[\faMale\faFemale] nos géniteurs, Bertin MEDEHOUIN et Antoinette ASSIBA, pour leur amour et leurs efforts inlassables en faveur de l’avenir de leurs enfants ; \vspace{0.1cm}	
	
	\item[\faUserGraduate] tous les camarades de la 36\textsuperscript{e} promotion des ISE de l’ENSEA, pour leur sens de la fraternité et leur soutien à tous égards ; \vspace{0.1cm}	
	
%	\item[\faMale \faFemale] Mademoiselle Espérance BALLOVI, pour son soutien moral.
\end{itemize}




\vspace{2cm}


\begin{flushright}
	{\small Crépin MEDEHOUIN} \\
	\vspace{0.2cm}
	{\small ENSEA d'Abidjan} \\
	
	\usdate\today
\end{flushright}

\end{SingleSpace}

	%
%	\clearemptydoublepage
	\pagestyle{plain}
%\chapter*{{Avant-Propos}\addcontentsline{toc}{chapter}{Avant-Propos}}




\chapter*{AVANT-PROPOS}
\addcontentsline{toc}{chapter}{AVANT-PROPOS}



\initial{L}’École Nationale Supérieure de Statistique et d’Économie Appliquée (ENSEA) d’Abidjan est un établissement public national dont la vocation est d’assurer la formation des statisticiens. Centre d’Excellence Africaine de la
Banque Mondiale, elle jouit d’une solide réputation dans ses domaines de compétences qui, du reste, lui sont spécifiques : Économie, Méthodes statistiques, notamment les méthodes quantitatives. Cette formation qui allie théorie et pratique, est délivrée à
travers des filières distinctes dont la division des Ingénieurs Statisticiens Économistes (ISE). \vspace{0.2cm}


Au cours de leur formation, les élèves ISE sont appelés à effectuer un stage d’application obligatoire de quatre mois à la fin de la troisième année aussi bien dans les institutions internationales que dans le secteur privé, public et parapublique. Ce
stage a essentiellement pour but de les aider à mettre en pratique des volets importants des enseignements théoriques reçus et à s’imprégner des réalités du milieu professionnel. Il est assorti de la rédaction d’un mémoire de stage.\vspace{0.2cm}


C’est dans ce cadre que nous avons été appelé à effectuer notre stage sur la période allant du 05 mai au 05 septembre 2025 au sein du Programme ANRS Coopération Côte d’Ivoire. Durant ce séjour, nous nous sommes, non seulement, imprégné du fonctionnement de cette structure, mais nous avons également eu la chance de travailler sur le thème : \og  Facteurs associés à la prise de poids chez les patients sous bithérapie dans l’essai MODERATO.\fg{} \vspace{0.2cm}


La réalisation de cette étude n’a pas connu d’entraves majeures. Elle a été facilitée par la disponibilité de notre encadrant et de tous les cadres de la structure qui n’ont ménagé aucun effort pour nous assurer un meilleur cadre et des conditions de travail optimales.



	\newpage 
	\renewcommand{\contentsname}{SOMMAIRE}
	\maxtocdepth{chapter}
	\tableofcontents
	%
% file: dedication.tex
% author: V?ctor Bre?a-Medina
% description: Contains the text for thesis dedication
%

\chapter*{SIGLES ET ACRONYMES}
\addcontentsline{toc}{chapter}{SIGLES ET ACRONYMES}



{
% Commande pour un style homogène
\newcommand{\sigle}[2]{\textbf{#1} & #2 \\}



\renewcommand{\arraystretch}{1.25} % Espacement vertical des lignes

\rowcolors{2}{gray!10}{white} % Alternance de couleurs de ligne

\begin{longtable}{>{\raggedright\arraybackslash}p{2cm} p{13cm}}
	\toprule
	\textbf{Sigle} & \textbf{Signification} \\
	\midrule
	\endfirsthead
	
	\toprule
	\textbf{Sigle} & \textbf{Signification} \\
	\midrule
	\endhead
	
	\sigle{AFD}{Agence Française de Développement}
	\sigle{ANRS}{Agence Nationale de Recherche sur le SIDA et les Hépatites Virales}
	\sigle{ARV}{Antirétroviral}
	\sigle{ATV/r}{Atazanavir/ritonavir}
	\sigle{CD4}{Lymphocytes T CD4+}
	\sigle{CV}{Charge Virale}
	\sigle{DTG}{Dolutégravir}
	\sigle{EFV}{Efavirenz}
	\sigle{ENSEA}{École Nationale Supérieure de Statistique et d’Économie Appliquée}
	\sigle{IMC}{Indice de Masse Corporelle}
	\sigle{INI}{Inhibiteur de l’Intégrase}
	\sigle{INTI}{Inhibiteur Nucléosidique de la Transcriptase Inverse}
	\sigle{INNTI}{Inhibiteur Non Nucléosidique de la Transcriptase Inverse}
	\sigle{IP}{Inhibiteur de la Protéase}
	\sigle{IRD}{Institut de Recherche pour le Développement}
	\sigle{OMS}{Organisation Mondiale de la Santé}
	\sigle{ONUSIDA}{Programme commun des Nations Unies sur le VIH/SIDA}
	\sigle{PAC-CI}{Programme ANRS Coopération Cote d’Ivoire}
	\sigle{PVVIH}{Personne Vivant avec le VIH}
	\sigle{SIDA}{Syndrome d’Immunodéficience Acquise}
	\sigle{TARV}{Traitement Antirétroviral}
	\sigle{TAF}{Ténofovir Alafénamide}
	\sigle{TDF}{Ténofovir Disoproxil Fumarate}
	\sigle{VIH}{Virus de l’Immunodéficience Humaine}
	\sigle{3TC}{Lamivudine}
	\bottomrule
\end{longtable}


}





%
%
%\begin{longtable}{@{}rrl@{}}
%	\textbf{AFRISTAT} & : & Observatoire Économique et Statistique d'Afrique Subsaharienne \\[0.5em]
%	\textbf{BIT}      & : & Bureau international du Travail \\[0.5em]
%	\textbf{CIV}      & : & Côte d'Ivoire \\[0.5em]
%	\textbf{CNUCED}   & : & Conférence des Nations Unies sur le Commerce et le Développement \\[0.5em]
%	\textbf{CIST}     & : & Conférence Internationale des Statisticiens du Travail \\[0.5em]
%	\textbf{EHCVM}    & : & Enquête Harmonisée sur les Conditions de Vie des Ménages \\[0.5em]
%	\textbf{EDS}      & : & Enquêtes Démographiques et de Santé \\[0.5em]
%	\textbf{EESI}     & : & Enquête sur l'Emploi et le Secteur Informel \\[0.5em]
%	\textbf{ERI-ESI}  & : & Enquête Régionale Intégrée sur l'Emploi et le Secteur Informel \\[0.5em]
%	\textbf{INS}      & : & Institut National de la Statistique \\[0.5em]
%	\textbf{IRD}      & : & Institut de Recherche et de Développement \\[0.5em]
%	\textbf{MI}       & : & Mobilité Intergénérationnelle \\[0.5em]
%	\textbf{OED}      & : & Origine Éducation Destination \\[0.5em]
%	\textbf{OIT}      & : & Organisation Internationale du Travail \\[0.5em]
%	\textbf{ONG}      & : & Organisation Non Gouvernementale \\[0.5em]
%	\textbf{PIB}      & : & Produit Intérieur Brut \\[0.5em]
%	\textbf{SCN}      & : & Système de Comptabilité Nationale \\[0.5em]
%	\textbf{UEMOA}    & : & Union économique et monétaire ouest-africaine \\[0.5em]
%	\textbf{UniDiff}  & : & Uniforme Différence \\[0.5em]
%\end{longtable}


\clearpage
	%
	\newpage
	\renewcommand{\listtablename}{LISTE DES TABLEAUX}
	\listoftables
	\addtocontents{lot}{}
	%
	\newpage
	\renewcommand{\listfigurename}{LISTE DES FIGURES}
	\listoffigures
	\addtocontents{lof}{}
	\newpage
	%
	\pagestyle{plain}


\chapter*{RÉSUME}
\addcontentsline{toc}{chapter}{RÉSUME}






\begin{SingleSpace}
	



\initial{D}ans cette étude, nous avons analysé et déterminé  les facteurs associés à la prise de poids chez des personnes vivant avec le VIH sous bithérapie, dans le cadre de l’essai MODERATO, le premier essai d’allègement thérapeutique mené en Afrique subsaharienne. L’étude, conduite en Côte d’Ivoire, au Burkina Faso et au Cameroun, a inclus 480 adultes dont 320 patients répartis aléatoirement dans les deux groupes de bi-thérapie : Dolutégravir + Lamivudine (DTG+3TC) et Atazanavir/ritonavir + Lamivudine (ATV/r+3TC), et suivis pendant 96 semaines. \vspace{0.2cm}


Pour répondre à notre problématique liée à la prise de poids des patients, nous avons appliqué deux approches statistiques. Un modèle linéaire mixte a permis d’analyser l’évolution du poids au fil du temps, tandis qu’un modèle de régression de Cox à risques proportionnels a été utilisé pour identifier les facteurs associés à un événement clinique défini par une prise de poids d’au moins 5\% par rapport au poids initial.\vspace{0.2cm}


Les résultats de l'étude ont montré que le régime à base de Dolutégravir (DTG+3TC) est associé à une prise de poids significativement plus importante que le régime à base d'Atazanavir/ritonavir (ATV/r+3TC). La vitesse de prise de poids était estimée à 553 grammes chaque 100 jours sous DTG+3TC, contre 360 grammes chaque 100 jours sous ATV/r+3TC, soit une différence annuelle d’environ 704,45 grammes. L’analyse multivariée du modèle de Cox a révélé trois principaux facteurs de risque de prise de poids ($\geqslant 5\%$). Premièrement, le sexe féminin, avec un risque 52\% plus élevé que celui des hommes. Deuxièmement, un faible poids initial, car chaque kilogramme supplémentaire réduit le risque de 2\%, ce qui confirme l’hypothèse d’un « retour à la santé » chez les patients initialement maigres. Enfin, la localisation géographique, les patients suivis au Burkina Faso ayant un risque 40\% plus élevé que ceux suivis en Côte d’Ivoire, ce qui suggère l’influence de facteurs locaux.\vspace{0.2cm}


Ces résultats soulignent l’importance d’une approche de médecine personnalisée pour les populations à risque, notamment les femmes et les patients de faible poids initial, surtout lorsqu’ils sont traités par DTG. Cette stratégie proactive est essentielle pour prévenir les complications cardiométaboliques à long terme tout en maintenant les excellents résultats virologiques de ces schémas thérapeutiques allégés.\\  \vspace{0.5cm}




\textbf{Mots-clés} : VIH, bithérapie, dolutégravir, atazanavir, prise de poids, Afrique de l'Ouest.



\end{SingleSpace}
	%
	
	


\chapter*{ABSTRACT}
\addcontentsline{toc}{chapter}{ABSTRACT}




\begin{SingleSpace}
	


\initial{I}n this study, we examined the factors associated with weight gain among people living with HIV receiving dual therapy, as part of the MODERATO trial, the first therapeutic simplification trial conducted in Africa. The study, carried out in Côte d’Ivoire, Burkina Faso, and Cameroon, included 320 patients randomly assigned to two groups: Dolutegravir + Lamivudine (DTG+3TC) and Atazanavir/ritonavir + Lamivudine (ATV/r+3TC), and followed over a period of 96 weeks. \vspace{0.2cm}

To address this issue, we applied two statistical approaches. A linear mixed model was used to analyze weight changes over time, while a Cox proportional hazards regression model was implemented to identify factors associated with a clinical event defined as a weight gain of at least 5\% from baseline. \vspace{0.2cm}

The study results showed that the Dolutegravir-based regimen (DTG+3TC) was associated with a significantly greater weight gain than the Atazanavir/ritonavir-based regimen (ATV/r+3TC). The rate of weight gain was estimated at 553 grams per 100 days under DTG+3TC, compared to 360 grams per 100 days under ATV/r+3TC, corresponding to an annual difference of approximately 704.45 grams. The multivariate Cox model analysis identified three main risk factors for rapid weight gain ($\geqslant 5\%$). First, female sex, with a 52\% higher risk compared to men. Second, low baseline weight, as each additional kilogram reduced the risk by 2\%, supporting the “return to health” hypothesis among initially underweight patients. Finally, geographical location: patients followed in Burkina Faso had a 40\% higher risk than those followed in Côte d’Ivoire, suggesting the influence of local factors.\vspace{0.2cm}

These findings highlight the importance of a personalized medicine approach, with enhanced metabolic monitoring for at-risk populations, particularly women and patients with low baseline weight, especially when treated with DTG. Such a proactive strategy is essential to prevent long-term cardiometabolic complications while maintaining the excellent virological outcomes of these simplified therapeutic regimens. \vspace{0.5cm}

\textbf{Keywords}: HIV, dual therapy, dolutegravir, atazanavir, weight gain, West Africa.


\end{SingleSpace}
\clearemptydoublepage


	% The bulk of the document is delegated to these chapter files in
	% subdirectories.
%	\input{frontmatter/	presentation_de_structure}
		


\chapter*{PRESENTATION DU CADRE DE STAGE}
\addcontentsline{toc}{chapter}{PRESENTATION DU CADRE DE STAGE}


\section*{1.1 \hspace{0.5cm}  Historique}


L’Agence Nationale de Recherche sur le SIDA et les Hépatites Virales (ANRS) a été fondée en 1992 par le gouvernement français dans le but de promouvoir et de coordonner la recherche sur le VIH/SIDA. Dans une dynamique d’internationalisation de la recherche, l’ANRS a établi plusieurs « sites de recherche » à l’étranger, en collaboration étroite avec les autorités sanitaires locales. Le programme PAC-CI / site ANRS Côte d’Ivoire a été mis en place en 1994, et officiellement structuré en 1996 par une convention multipartite signée entre le ministère ivoirien en charge de la Santé, le ministère de l’Économie et des Finances, le ministère français de la Coopération et l’ANRS. Cette convention fixait deux objectifs majeurs :


\begin{itemize}[label=\textbullet] % classique
	
	\item La formation du personnel de santé aux méthodes de recherche médicale appliquées au VIH/SIDA ;
	
	\item La conduite de recherches médicales visant à produire des résultats directement utiles aux personnes vivant avec le virus.
	
\end{itemize}

Initialement constituée sous forme de Groupement d’Intérêt Public (GIP), l’agence a été intégrée en janvier 2012 à l’Institut National de la Santé et de la Recherche Médicale (Inserm), tout en conservant une autonomie fonctionnelle. Jusqu’en 2023, huit sites ANRS étaient actifs dans le monde, notamment au Brésil, au Burkina Faso, au Cambodge, au Cameroun, en Côte d’Ivoire, en Égypte, au Sénégal et au Vietnam.



\section*{1.2 \hspace{0.5cm} Cadre constitutionnel actuel }

En 2010, les partenaires ivoiriens et français, estimant le bilan du programme PAC-CI positif, ont décidé de renforcer leur collaboration en révisant la convention initiale de 1996. Cette révision, signée en février 2010, a élargi le cercle des signataires. Côté ivoirien, elle a été cosignée par le ministère de l’Enseignement supérieur et de la Recherche scientifique, le ministère de la Santé et de la Lutte contre le SIDA, ainsi que le ministère de l’Économie et des Finances. Du côté français, l’Inserm, l’ANRS, l’Université de Bordeaux 2 et l’Ambassade de France en Côte d’Ivoire y ont pris part. Par ailleurs, cette nouvelle convention a étendu les missions du programme PAC-CI, dont l’objectif ne se limitait plus à la recherche sur le VIH/SIDA, mais s’ouvrait désormais à d’autres maladies infectieuses.\\

En janvier 2023, le programme PAC-CI a été signataire de «PRISME-CI» : Plateforme de recherche en santé mondiale. La nouvelle convention cadre de PRISME-CI se situe clairement dans la continuité des deux conventions PAC-CI précédentes (1996 et 2010), mais fait évoluer le programme sur trois points importants. D’abord, la plateforme PRISME-CI adopte une orientation résolument axée sur la Santé Mondiale, traduisant la volonté conjointe des partenaires français et ivoiriens d’élargir leur stratégie à une approche globale, interdisciplinaire et mondialisée, notamment en intégrant la perspective « One Health ». Bien que les maladies infectieuses restent au cœur du programme, une ouverture progressive vers certaines pathologies non infectieuses est envisagée. Ensuite, l’Institut de Recherche pour le Développement (IRD) rejoint officiellement les partenaires du programme, renforçant ainsi sa dimension scientifique et internationale. Enfin, la convention PRISME-CI consolide la plateforme collaborative d’ingénierie scientifique Bordeaux-Abidjan, connue sous le nom de MEREVA. Déjà à l’origine de nombreux succès du programme PAC-CI, cette plateforme experte dans le montage et la gestion de projets de recherche multinationaux devient désormais une structure ouverte et visible à l’extérieur, après avoir longtemps fonctionné comme un outil technique interne.









	\newpage 
	\clearemptydoublepage
	\pagenumbering{arabic}
\mainmatter
% Réinitialisation explicite après le début du corps principal
\setcounter{figure}{0}
\setcounter{table}{0}
\import{chapters/}{Introduction.tex}

	\import{chapters/chapter01/}{Revue_litterature.tex}
	\newpage
	\import{chapters/Chapter02/}{Donnee_Methodologie.tex}
	\newpage
%	\import{chapters/Chapter03/}{Descriptible.tex}
%	\newpage
	\import{chapters/Chapter04/}{Resultats.tex}
	\newpage
	\import{chapters/}{Conclusion.tex}
	%\pagestyle{plain}
\chapter*{CONCLUSION}\addcontentsline{toc}{chapter}{CONCLUSION}


En définitive, cette étude s’inscrit dans le cadre de l’essai clinique MODERATO, dont l’objectif était d’évaluer l’efficacité de stratégies thérapeutiques allégées chez les personnes vivant avec le VIH en Afrique de l’Ouest et centrale. Ainsi, notre travail s’est concentré sur les facteurs pouvant influencer le poids des patients dans cet essai ; plus précisément ceux traités par bithérapie : le dolutégravir + lamivudine (DTG+3TC) et l’atazanavir/ritonavir + lamivudine (ATV/r+3TC).\vspace{0.2cm}

L’analyse a été menée à partir de données longitudinales issues d’un échantillon de 320 patients, répartis de manière équitable entre les deux bras thérapeutiques. Le suivi s’est étendu sur 96 semaines, permettant une évaluation rigoureuse de l’évolution pondérale sous bithérapie. Afin de tirer pleinement parti de la richesse de ces données, plusieurs méthodes statistiques complémentaires ont été mobilisées. Le modèle linéaire mixte a permis d’étudier les trajectoires individuelles de poids au cours du temps et le modèle de régression de Cox multivarié a permis d’identifier les déterminants associés à une prise d'au moins 5\%. Ces différentes approches ont convergé vers des résultats cohérents et informatifs. \vspace{0.2cm}


L’analyse des résultats permet de dégager deux constats essentiels. Tout d’abord, les patients traités par la combinaison DTG+3TC ont montré une tendance à prendre davantage de poids que ceux sous ATV/r+3TC. Toutefois, cette différence, bien que présente, s’est révélée plus modérée que ce qui a été observé dans certaines études, suggérant une spécificité contextuelle des effets métaboliques dans notre contexte. Ensuite, plusieurs caractéristiques individuelles se sont avérées fortement liées à la prise de poids. Les femmes, les patients présentant un poids initial relativement faible et ainsi que les patients traités au Burkina, ont été susceptible de prise pondérale d'au moins 5\%. Ces résultats confirment l’influence de facteurs démographiques et cliniques sur les réponses au traitement antirétroviral.  \vspace{0.2cm}


Ces résultats revêtent une portée clinique importante pour la prise en charge des personnes vivant avec le VIH dans les pays d’Afrique subsaharienne. Ils plaident en faveur d’un suivi métabolique renforcé, en particulier chez les femmes, les patients présentant un poids initial relativement faible et surtout s'ils sont traités par dolutégravir ; cette dernière molécule étant le traitement de choix recommandé par l’OMS en tri comme en bi-thérapie. Des actions de prévention, incluant un accompagnement nutritionnel adapté et la promotion de l’activité physique, pourraient contribuer à limiter les effets métaboliques indésirables associés au traitement. \newpage

En somme, cette étude apporte des éléments de réflexion essentiels pour une optimisation thérapeutique contextualisée et individualisée. Elle met en lumière l’importance d’une approche personnalisée du traitement antirétroviral, articulant efficacité virologique, tolérance métabolique et qualité de vie. Ces résultats s’inscrivent dans une perspective de médecine de précision et de prévention, adaptée aux réalités cliniques et épidémiologiques des pays africains.
%
%
%
%
%\section*{Recommandations}
%
%À la lumière de ces résultats, plusieurs recommandations peuvent être formulées à destination des cliniciens, chercheurs et décideurs en santé publique : \\
%
%\textbf{Suivi métabolique renforcé des femmes et des patients âgés :} Ces populations présentent un risque accru de prise de poids significative sous bithérapie. Une surveillance régulière du poids, de la composition corporelle et des paramètres métaboliques est donc indispensable. \vspace{0.2cm}
%	
% \textbf{Intégration du poids initial dans la décision thérapeutique :} Les patients ayant un faible poids à l’inclusion devraient faire l’objet d’un suivi nutritionnel spécifique, afin de différencier un gain pondéral bénéfique d’une évolution vers le surpoids ou l’obésité. \vspace{0.2cm}
%	
%	
%\textbf{Renforcement de la recherche en contexte africain :} Les effets métaboliques des ARV, largement documentés en Occident, doivent être étudiés plus systématiquement en Afrique subsaharienne où les facteurs nutritionnels, génétiques et socioéconomiques peuvent modifier les trajectoires de poids.
%	

%
%Enfin, il serait pertinent que les futures recommandations cliniques intègrent la dimension métabolique comme critère de choix des schémas thérapeutiques, en particulier dans les stratégies d’allègement à visée long terme. La mise en œuvre de ces recommandations contribuera à améliorer la qualité de vie des patients, tout en consolidant les acquis thérapeutiques dans la lutte contre le VIH.
\newpage


	\pagenumbering{roman}
\setcounter{page}{13}  % ← à définir manuellement (ex : xi = 11)
        \backmatter
        \printbibliography[title={BIBLIOGRAPHIE}]  % Définit le titre directement
	
%	\clearemptydoublepage
	%
	
	%\cleardoublepage



	\import{chapters/appendices/}{AnnexeA.tex}
%\import{chapters/appendices/}{AnnexeB.tex}
	\newpage
	\renewcommand{\contentsname}{TABLE DES MATIERES}
	\setcounter{tocdepth}{4}
	\tableofcontents
	
	
\end{document}