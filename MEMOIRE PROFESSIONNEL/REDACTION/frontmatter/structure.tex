


\chapter*{PRESENTATION DU CADRE DE STAGE}
\addcontentsline{toc}{chapter}{PRESENTATION DU CADRE DE STAGE}


\section*{1.1 \hspace{0.5cm}  Historique}


L’Agence Nationale de Recherche sur le SIDA et les Hépatites Virales (ANRS) a été fondée en 1992 par le gouvernement français dans le but de promouvoir et de coordonner la recherche sur le VIH/SIDA. Dans une dynamique d’internationalisation de la recherche, l’ANRS a établi plusieurs « sites de recherche » à l’étranger, en collaboration étroite avec les autorités sanitaires locales. Le programme PAC-CI / site ANRS Côte d’Ivoire a été mis en place en 1994, et officiellement structuré en 1996 par une convention multipartite signée entre le ministère ivoirien en charge de la Santé, le ministère de l’Économie et des Finances, le ministère français de la Coopération et l’ANRS. Cette convention fixait deux objectifs majeurs :


\begin{itemize}[label=\textbullet] % classique
	
	\item La formation du personnel de santé aux méthodes de recherche médicale appliquées au VIH/SIDA ;
	
	\item La conduite de recherches médicales visant à produire des résultats directement utiles aux personnes vivant avec le virus.
	
\end{itemize}

Initialement constituée sous forme de Groupement d’Intérêt Public (GIP), l’agence a été intégrée en janvier 2012 à l’Institut National de la Santé et de la Recherche Médicale (Inserm), tout en conservant une autonomie fonctionnelle. Jusqu’en 2023, huit sites ANRS étaient actifs dans le monde, notamment au Brésil, au Burkina Faso, au Cambodge, au Cameroun, en Côte d’Ivoire, en Égypte, au Sénégal et au Vietnam.



\section*{1.2 \hspace{0.5cm} Cadre constitutionnel actuel }

En 2010, les partenaires ivoiriens et français, estimant le bilan du programme PAC-CI positif, ont décidé de renforcer leur collaboration en révisant la convention initiale de 1996. Cette révision, signée en février 2010, a élargi le cercle des signataires. Côté ivoirien, elle a été cosignée par le ministère de l’Enseignement supérieur et de la Recherche scientifique, le ministère de la Santé et de la Lutte contre le SIDA, ainsi que le ministère de l’Économie et des Finances. Du côté français, l’Inserm, l’ANRS, l’Université de Bordeaux 2 et l’Ambassade de France en Côte d’Ivoire y ont pris part. Par ailleurs, cette nouvelle convention a étendu les missions du programme PAC-CI, dont l’objectif ne se limitait plus à la recherche sur le VIH/SIDA, mais s’ouvrait désormais à d’autres maladies infectieuses.\\

En janvier 2023, le programme PAC-CI a été signataire de «PRISME-CI» : Plateforme de recherche en santé mondiale. La nouvelle convention cadre de PRISME-CI se situe clairement dans la continuité des deux conventions PAC-CI précédentes (1996 et 2010), mais fait évoluer le programme sur trois points importants. D’abord, la plateforme PRISME-CI adopte une orientation résolument axée sur la Santé Mondiale, traduisant la volonté conjointe des partenaires français et ivoiriens d’élargir leur stratégie à une approche globale, interdisciplinaire et mondialisée, notamment en intégrant la perspective « One Health ». Bien que les maladies infectieuses restent au cœur du programme, une ouverture progressive vers certaines pathologies non infectieuses est envisagée. Ensuite, l’Institut de Recherche pour le Développement (IRD) rejoint officiellement les partenaires du programme, renforçant ainsi sa dimension scientifique et internationale. Enfin, la convention PRISME-CI consolide la plateforme collaborative d’ingénierie scientifique Bordeaux-Abidjan, connue sous le nom de MEREVA. Déjà à l’origine de nombreux succès du programme PAC-CI, cette plateforme experte dans le montage et la gestion de projets de recherche multinationaux devient désormais une structure ouverte et visible à l’extérieur, après avoir longtemps fonctionné comme un outil technique interne.







