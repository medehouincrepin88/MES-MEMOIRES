\pagestyle{plain}


\chapter*{RÉSUME}
\addcontentsline{toc}{chapter}{RÉSUME}






\begin{SingleSpace}
	



\initial{D}ans cette étude, nous avons analysé et déterminé  les facteurs associés à la prise de poids chez des personnes vivant avec le VIH sous bithérapie, dans le cadre de l’essai MODERATO, le premier essai d’allègement thérapeutique mené en Afrique subsaharienne. L’étude, conduite en Côte d’Ivoire, au Burkina Faso et au Cameroun, a inclus 480 adultes dont 320 patients répartis aléatoirement dans les deux groupes de bi-thérapie : Dolutégravir + Lamivudine (DTG+3TC) et Atazanavir/ritonavir + Lamivudine (ATV/r+3TC), et suivis pendant 96 semaines. \vspace{0.2cm}


Pour répondre à notre problématique liée à la prise de poids des patients, nous avons appliqué deux approches statistiques. Un modèle linéaire mixte a permis d’analyser l’évolution du poids au fil du temps, tandis qu’un modèle de régression de Cox à risques proportionnels a été utilisé pour identifier les facteurs associés à un événement clinique défini par une prise de poids d’au moins 5\% par rapport au poids initial.\vspace{0.2cm}


Les résultats de l'étude ont montré que le régime à base de Dolutégravir (DTG+3TC) est associé à une prise de poids significativement plus importante que le régime à base d'Atazanavir/ritonavir (ATV/r+3TC). La vitesse de prise de poids était estimée à 553 grammes chaque 100 jours sous DTG+3TC, contre 360 grammes chaque 100 jours sous ATV/r+3TC, soit une différence annuelle d’environ 704,45 grammes. L’analyse multivariée du modèle de Cox a révélé trois principaux facteurs de risque de prise de poids ($\geqslant 5\%$). Premièrement, le sexe féminin, avec un risque 52\% plus élevé que celui des hommes. Deuxièmement, un faible poids initial, car chaque kilogramme supplémentaire réduit le risque de 2\%, ce qui confirme l’hypothèse d’un « retour à la santé » chez les patients initialement maigres. Enfin, la localisation géographique, les patients suivis au Burkina Faso ayant un risque 40\% plus élevé que ceux suivis en Côte d’Ivoire, ce qui suggère l’influence de facteurs locaux.\vspace{0.2cm}


Ces résultats soulignent l’importance d’une approche de médecine personnalisée pour les populations à risque, notamment les femmes et les patients de faible poids initial, surtout lorsqu’ils sont traités par DTG. Cette stratégie proactive est essentielle pour prévenir les complications cardiométaboliques à long terme tout en maintenant les excellents résultats virologiques de ces schémas thérapeutiques allégés.\\  \vspace{0.5cm}




\textbf{Mots-clés} : VIH, bithérapie, dolutégravir, atazanavir, prise de poids, Afrique de l'Ouest.



\end{SingleSpace}