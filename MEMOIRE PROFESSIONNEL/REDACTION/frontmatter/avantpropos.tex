\pagestyle{plain}
%\chapter*{{Avant-Propos}\addcontentsline{toc}{chapter}{Avant-Propos}}




\chapter*{AVANT-PROPOS}
\addcontentsline{toc}{chapter}{AVANT-PROPOS}



\initial{L}’École Nationale Supérieure de Statistique et d’Économie Appliquée (ENSEA) d’Abidjan est un établissement public national dont la vocation est d’assurer la formation des statisticiens. Centre d’Excellence Africaine de la
Banque Mondiale, elle jouit d’une solide réputation dans ses domaines de compétences qui, du reste, lui sont spécifiques : Économie, Méthodes statistiques, notamment les méthodes quantitatives. Cette formation qui allie théorie et pratique, est délivrée à
travers des filières distinctes dont la division des Ingénieurs Statisticiens Économistes (ISE). \vspace{0.2cm}


Au cours de leur formation, les élèves ISE sont appelés à effectuer un stage d’application obligatoire de quatre mois à la fin de la troisième année aussi bien dans les institutions internationales que dans le secteur privé, public et parapublique. Ce
stage a essentiellement pour but de les aider à mettre en pratique des volets importants des enseignements théoriques reçus et à s’imprégner des réalités du milieu professionnel. Il est assorti de la rédaction d’un mémoire de stage.\vspace{0.2cm}


C’est dans ce cadre que nous avons été appelé à effectuer notre stage sur la période allant du 05 mai au 05 septembre 2025 au sein du Programme ANRS Coopération Côte d’Ivoire. Durant ce séjour, nous nous sommes, non seulement, imprégné du fonctionnement de cette structure, mais nous avons également eu la chance de travailler sur le thème : \og  Facteurs associés à la prise de poids chez les patients sous bithérapie dans l’essai MODERATO.\fg{} \vspace{0.2cm}


La réalisation de cette étude n’a pas connu d’entraves majeures. Elle a été facilitée par la disponibilité de notre encadrant et de tous les cadres de la structure qui n’ont ménagé aucun effort pour nous assurer un meilleur cadre et des conditions de travail optimales.


