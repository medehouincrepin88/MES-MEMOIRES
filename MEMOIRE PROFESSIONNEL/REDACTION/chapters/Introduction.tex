\pagestyle{plain}
\chapter*{INTRODUCTION}\addcontentsline{toc}{chapter}{INTRODUCTION}


\section*{Contexte et justification}

\initial{D}écouvert en 1983 par des chercheurs français de l’Institut Pasteur \parencite{barre1983isolation}, le virus de l’immunodéficience humaine (VIH) s’est rapidement imposé comme l’un des plus grands défis sanitaires du XXe siècle. Dès les premières années, l’épidémie a révélé son caractère multidimensionnel, touchant non seulement la santé publique, mais aussi les sphères sociale, économique et politique à l’échelle mondiale. Face à une maladie incurable et hautement stigmatisée, la communauté scientifique s'est mobilisée pour étudier les mécanismes de l’infection, faciliter son dépistage et développer des traitements. En 1985, l’azidothymidine, un inhibiteur de la transcriptase inverse, marque une avancée majeure en devenant le premier traitement contre le VIH \parencite{mitsuya19853}.  Bien que prometteuse, cette monothérapie à l’azidothymidine présentait une efficacité limitée pour contrôler durablement l’infection et, à fortes doses, entraînait souvent des effets secondaires graves chez les patients \parencite{fischl1987efficacy}. Néanmoins, cette avancée a jeté les bases pour de nouvelles classes thérapeutiques, parmi lesquelles figurent notamment les inhibiteurs non nucléosidiques de la transcriptase inverse et les inhibiteurs de protéase, intervenant à différents stades du cycle viral pour en inhiber la prolifération. À partir de 1995, le traitement du VIH va s'appuyer sur une approche par trithérapie associant deux inhibiteurs nucléosidiques de la transcriptase inverse avec un inhibiteur de protéase ou un inhibiteur non nucléosidique \parencite{katzenstein1997antiretroviral}. Officiellement, l’Organisation mondiale de la Santé (OMS) formule en 2004 des recommandations, actualisées en 2006, établissant la trithérapie comme traitement de première intention contre le VIH \parencite{WHO2006ART}.\vspace{0.3cm}

Aujourd’hui, la pharmacopée antirétrovirale, enrichie depuis 2007 par l’introduction des inhibiteurs d’intégrase \parencite{lefeuvre2022acces}, propose des options thérapeutiques variées, alliant simplicité et meilleure tolérance \parencite{hal03920940}. Ces progrès ont transformé le pronostic du VIH, permettant un contrôle durable de la charge virale et faisant de cette infection, dans de nombreux contextes, une maladie chronique avec laquelle il est possible de vivre. Selon les dernières estimations de l'Organisation des Nations Unies pour la lutte contre le Sida (ONUSIDA) et de l’OMS publiées en 2023, le nombre de décès liés au SIDA a été réduit de 69 \% depuis le pic de 2004 et on estime également une baisse de nouvelles infections au VIH  survenues à travers le monde en 2022 de 53\% par rapport au pic de 2,8 millions atteint en 1997 \parencite{ONUSIDA2023}. Ces chiffres s’inscrivent dans la dynamique portée par l’ONUSIDA, qui vise à éliminer le sida comme menace pour la santé publique d’ici 2030. L'objectif repose sur la stratégie dite « 95-95-95 », qui vise à ce que 95\% des personnes vivant avec le VIH connaissent leur statut sérologique, que 95\% de celles-ci bénéficient d’un traitement antirétroviral, et que 95\% des patients traités présentent une charge virale indétectable.\vspace{0.3cm}




Malgré ces avancées thérapeutiques majeures, les antirétroviraux, bien qu’efficaces contre la réplication virale, ne parviennent pas à éradiquer le virus en raison de son intégration dans les réservoirs cellulaires. Le traitement antirétroviral doit être poursuivi à vie pour maintenir une charge virale indétectable, prévenir la progression vers le sida, et limiter le risque de transmission. Cette chronicité du traitement impose une observance rigoureuse et une attention particulière à la tolérance à long terme \parencite{palich2021traitement}. Face à ces défis, l’avenir du traitement s’oriente vers des schémas thérapeutiques allégés, moins toxiques et plus simples à suivre, notamment à travers des régimes de bi-thérapie, des formes à libération prolongée ou injectables. \vspace{0.3cm}


À ce jour, les avancées récentes en matière d’allègement thérapeutique ont conduit à plusieurs options scientifiquement validées. Parmi celles-ci, les formes injectables intramusculaires à administration mensuelle ou bimestrielle ont démontré leur efficacité clinique \parencite{swindells2020long}, tout comme les stratégies d’administration intermittente (4 à 5 jours par semaine), actuellement en cours d’évaluation et susceptibles d’être intégrées aux futures recommandations thérapeutiques \parencite{palich2021traitement}. De même, diverses bithérapies combinant un inhibiteur de protéase boosté (IP/r) ou un inhibiteur d’intégrase (INI) avec un inhibiteur nucléosidique ou non nucléosidique de la transcriptase inverse (INTI ou INNTI) ont prouvé leur efficacité et sont désormais reconnues dans les recommandations internationales \parencite{van2020efficacy, llibre2018efficacy}. Bien que ces stratégies innovantes connaissent une large adoption dans les pays du Nord, leur mise en œuvre reste limitée dans les pays du Sud. C’est donc dans le cadre de l’évaluation de l’efficacité des bithérapies dans notre contexte africain que s’inscrit l’essai clinique MODERATO, premier essai d’allègement en bithérapie conduit sur le continent africain \footnote{\href{https://vih.org/vih-et-sante-sexuelle/20250722/premier-essai-dallegement-en-bitherapie-en-afrique-essai-anrs-moderato/}{https://vih.org/vih-et-sante-sexuelle/20250722/premier-essai-dallegement-en-bitherapie-en-afrique-essai-anrs-moderato/}}. Mené en Afrique de l’Ouest et centrale (Burkina Faso, Cameroun, Côte d’Ivoire), cet essai clinique randomisé de non-infériorité avait pour objectif de vérifier si une bithérapie de maintenance (DTG+3TC ou ATV/r+3TC) était aussi efficace, après 96 semaines, que la poursuite de la trithérapie de référence de l’OMS (TDF+3TC+EFV ou DTG+3TC+TDF) chez des patients déjà en succès virologique depuis au moins deux ans. \vspace{0.3cm}


Cependant, bien que cette stratégie d’allègement thérapeutique présente des avantages significatifs en termes de simplification des schémas posologiques et d’amélioration de la tolérance, elle soulève également de nouvelles préoccupations. Un effet indésirable émergent a été observé ces dernières années au sein de la classe des inhibiteurs d’intégrase (INI), notamment avec le dolutégravir (DTG) : une prise de poids excessive et inattendue durant le traitement \parencite{menard2017dolutegravir, waters2023limited}. Les conséquences de cette prise de poids sur les risques métaboliques et cardiovasculaires restent encore mal établies \parencite{bourgi2020greater}, alimentant un débat entre deux interprétations : d’un côté, celle d’un simple « retour à la santé », et de l’autre, celle d’un effet secondaire potentiellement néfaste  pouvant accroître le risque cardiovasculaire.   Il convient toutefois de souligner que le DTG est généralement prescrit dans le cadre de traitements combinés incluant d’autres antirétroviraux, dont l’implication dans la prise de poids reste encore insuffisamment explorée \parencite{taramasso2020factors}. Par ailleurs, bien que les mécanismes physiopathologiques sous-jacents à cette prise de poids demeurent mal élucidés, certaines caractéristiques propres aux personnes vivant avec le VIH (PVVIH) ont déjà été associés à une prise de poids : un faible taux de CD4
\parencite{venter2019dolutegravir, bakal2018obesity}, une charge virale élevée \parencite{venter2019dolutegravir, sax2020weight}, un indice de masse corporelle (IMC) initial faible \parencite{taramasso2017weight} ainsi que le sexe féminin \parencite{venter2019dolutegravir,  sax2020weight,  bakal2018obesity}. \vspace{0.3cm}

Ainsi, comprendre les facteurs associés à cette prise de poids dans notre étude est essentiel pour garantir une prise en charge équilibrée, conciliant efficacité, tolérance et qualité de vie. La question centrale de cette étude peut dès lors être formulée ainsi : Quels sont les déterminants de la prise de poids chez les patients traités par bithérapie dans le cadre de l’essai MODERATO ? 
Pour cerner l’ensemble des éléments autour de cette préoccupation, les questions de recherche formulées dans le cadre de notre étude sont les suivantes : %\vspace{0.2cm}

\begin{itemize}[label=\textbullet] % classique
	
	\item La prise de poids observée diffère-t-elle selon le type de bithérapie administrée (DTG+3TC vs. ATV/r+3TC) ? 
	
	\item  Quels facteurs individuels et cliniques expliquent les variations pondérales chez les patients recevant soit DTG+3TC, soit ATV/r+3TC ?
\end{itemize}

\section*{Objectifs de l'étude}
L'objectif général est d'étudier les facteurs associés à la prise de poids observée chez les patients vivant avec le VIH recevant la bithérapie dans l’essai clinique MODERATO. \vspace{0.5cm}



\noindent De façon spécifique, nous cherchons à  :  \vspace{0.2cm}
\begin{itemize}
	
	\item[$OBS_1$] : Comparer la prise de poids entre les différents bras de bithérapie (DTG/3TC et ATV/r + 3TC) ;
	
	\item[$OBS_2$] : Identifier les facteurs de risque associés à un gain de poids d’au moins 5\%.
\end{itemize}

\newpage
\section*{Hypothèses}

\begin{itemize}
	\item[$H_1$] : Les patients sous DTG + 3TC présentent une augmentation de poids significativement plus marquée que ceux traités par ATV/r + 3TC ; 
	
	\item[$H_2$] : Les femmes présentent une probabilité significativement plus élevée de connaître un gain de poids d’au moins 5\% par rapport aux hommes. 
	
\end{itemize}



\section*{Annonce du plan}
En vue d’atteindre les objectifs fixés, ce document est structuré en trois chapitres. Le premier chapitre introduit les concepts fondamentaux ainsi que le cadre empirique en lien avec la problématique étudiée. Le deuxième chapitre détaille la source des données mobilisées et la méthodologie adoptée pour l’analyse. Le troisième chapitre, quant à lui, présente les caractéristiques des patients inclus dans l’échantillon, expose les résultats empiriques issus des différents modèles estimés et propose une discussion.
















