%
% File: chap01.tex
% Author: Victor F. Brena-Medina
% Description: Introduction chapter where the biology goes.
%


\let\textcircled=\pgftextcircled
\chapter{APPROCHE CONCEPTUELLE ET REVUE DE LITTÉRATURE}
\label{chap:revue}



\section{Définitions des concepts clés} 
Dans cette section, les concepts clés de l’étude sont définis pour faciliter la compréhension du travail. 



\subsection*{VIH / SIDA}
\textbf{Le Virus de l’Immunodéficience Humaine (VIH)} : est un rétrovirus responsable de l’infection qui conduit à une dégradation progressive du système immunitaire humain. Ce virus attaque principalement les lymphocytes T CD4+, des cellules essentielles à la réponse immunitaire. Sans traitement, l’infection par le VIH affaiblit progressivement la capacité de l’organisme à lutter contre les infections et certaines formes de cancer.

\textbf{Le Syndrome d’Immunodéficience Acquise (SIDA)} : désigne le stade avancé de l’infection à VIH, caractérisé par une chute sévère du nombre de lymphocytes CD4+ et l’apparition de maladies opportunistes ou de cancers liés à l’immunodépression. Le SIDA constitue la phase terminale de l’infection, qui peut être retardée ou évitée grâce à la mise en œuvre précoce et prolongée d’un traitement antirétroviral (TARV).



\subsection*{Lymphocytes T CD4}
Les cellules CD4, aussi appelées lymphocytes T CD4, sont des globules blancs essentiels à la coordination de la réponse immunitaire. Le VIH cible spécifiquement ces cellules et les détruit progressivement, affaiblissant ainsi les défenses naturelles de l'organisme. Le nombre de CD4 par millimètre cube de sang (taux de CD4) est un indicateur important pour évaluer la progression de l'infection, déterminer le moment d’initier ou d’ajuster le traitement antirétroviral, et surveiller l’état immunitaire du patient.



\subsection*{Charge virale}
La charge virale correspond à la quantité de virus VIH présente dans le sang d’une personne infectée, généralement exprimée en copies d’ARN viral par millilitre (copies/ml). Elle constitue un indicateur central pour évaluer la progression de l’infection et l’efficacité du traitement antirétroviral (TARV). 



\subsection*{Traitement antirétroviral (TARV)}
Le traitement antirétroviral (TARV) désigne l’ensemble des médicaments utilisés pour contrôler l’infection par le virus de l’immunodéficience humaine (VIH). Il vise à supprimer la réplication virale, restaurer et préserver la fonction immunitaire, réduire la morbidité et la mortalité liées au VIH, ainsi que prévenir la transmission du virus.  Ces traitements sont administrés de façon prolongée, voire à vie, et leur optimisation est un enjeu majeur de santé publique.


\subsection*{Contrôle virologique / Suppression virologique}
Il s'agit de la réduction de la charge virale du VIH dans le sang à un niveau indétectable grâce à un traitement antirétroviral efficace. Cette suppression est généralement atteinte lorsque le nombre de copies du virus par millilitre de sang est inférieur au seuil de détection des tests disponibles (souvent < 50 copies/ml). 


%
%\subsection*{Allègement thérapeutique}
%L’allègement thérapeutique désigne une approche consistant à réduire l’intensité ou le nombre de médicaments dans le traitement antirétroviral (TARV), tout en maintenant l’efficacité du contrôle de la réplication virale. Cette stratégie est envisagée chez des patients vivant avec le VIH dont la charge virale est durablement supprimée et qui présentent une bonne observance au traitement.
%L’objectif principal est de diminuer les effets indésirables à long terme, de réduire les coûts, d'améliorer la qualité de vie des patients, et de limiter la toxicité cumulative des traitements, sans compromettre le succès thérapeutique. L’allègement peut prendre plusieurs formes : bithérapies, monothérapies ou espacements des prises.








\subsection*{Stratégie de maintenance}
Une stratégie de maintenance s’inscrit dans une logique de continuité des soins chez les patients vivant avec le VIH après l’atteinte d’une suppression virologique efficace sous TARV. Elle vise à maintenir cette suppression à long terme avec un schéma thérapeutique optimisé, souvent plus léger ou mieux toléré. 

\subsection*{Prise de poids sous antirétroviraux}

La prise de poids sous antirétroviraux désigne l’augmentation du poids corporel observée chez les personnes vivant avec le VIH après le début du traitement. Elle peut être bénéfique lorsqu’elle permet de retrouver un poids compatible avec une bonne santé. Cependant, lorsqu’elle dépasse certains seuils définis par l’indice de masse corporelle (IMC) ou par des gains pondéraux spécifiques, elle devient un facteur de risque pour des troubles métaboliques et cardiovasculaires. Par exemple, un gain de poids supérieur à 5-10\% du poids initial ou un passage dans la catégorie surpoids (IMC $\geq$ 25 kg/m²) peut alerter sur un risque accru.



\newpage



%              Ando et al. (2021)

%L'étude d'Ando et al. (2021) visait à évaluer les changements de poids à long terme chez des patients asiatiques naïfs de traitement vivant avec le VIH (PVVIH) après l'initiation d'une thérapie antirétrovirale (TAR). L'objectif principal était d'analyser les variations de poids sur une période de cinq ans en fonction des différentes classes d'antirétroviraux, notamment les inhibiteurs de l'intégrase (INSTI), les inhibiteurs de protéase (IP) et les inhibiteurs non nucléosidiques de la transcriptase inverse (INNTI), ainsi que des combinaisons spécifiques de médicaments. L'échantillon comprenait 1 579 patients asiatiques adultes naïfs de traitement, recrutés entre janvier 2005 et février 2019 dans un centre clinique spécialisé à Tokyo, et suivis jusqu'en octobre 2019. Les données ont été collectées de manière rétrospective, avec des mesures de poids, de charge virale et de compte de cellules CD4 effectuées tous les trois mois. Les auteurs ont utilisé des modèles linéaires à effets mixtes multivariés pour prédire les changements de poids, en ajustant pour des facteurs tels que l'âge, le sexe, l'année d'initiation du TAR, le poids initial, la charge virale et le compte de CD4. Les principaux résultats ont montré que les patients sous dolutégravir (DTG) combiné au ténofovir alafénamide (TAF)/emtricitabine (FTC) présentaient le gain de poids le plus important, avec une augmentation moyenne de 6,7 kg après cinq ans, suivis par ceux sous DTG seul (5,3 kg), darunavir (DRV, 4,1 kg) et élvitégravir (EVG, 4,6 kg). En revanche, les régimes à base de raltégravir (RAL), lopinavir (LPV) et atazanavir (ATV) étaient associés à des gains de poids moindres (1,9 kg, 2,1 kg et 2,3 kg respectivement). Les médicaments de base comme le TAF étaient également liés à un gain de poids plus élevé comparé au ténofovir disoproxil fumarate (TDF) ou à l'abacavir (ABC). Les discussions ont souligné que ces résultats concordent avec des études antérieures menées sur d'autres populations, bien que les patients asiatiques semblent prendre moins de poids que les non-Asiatiques. Les auteurs ont également noté des limites, notamment le biais de sélection dû au design monocentrique et la prédominance masculine de l'échantillon. Ils concluent que le DTG combiné au TAF/FTC est associé au gain de poids le plus significatif chez les PVVIH asiatiques, tout en recommandant une vigilance accrue pour les risques métaboliques à long terme.

%**Référence :**  
%Ando, N., Nishijima, T., Mizushima, D., Inaba, Y., Kawasaki, Y., Kikuchi, Y., Oka, S., & Gatanaga, H. (2021). Long-term weight gain after initiating combination antiretroviral therapy in treatment-naïve Asian people living with human immunodeficiency virus. *International Journal of Infectious Diseases, 110*, 21–28. https://doi.org/10.1016/j.ijid.2021.07.030



%                     Bourgi et al. (2020)

%L'étude de Bourgi et al. (2020) visait à comparer les différences de gain de poids à court terme chez des patients naïfs de traitement vivant avec le VIH (PVVIH) après l'initiation de différentes thérapies antirétrovirales (TAR), en mettant l'accent sur les inhibiteurs de l'intégrase (INSTI), notamment le dolutégravir (DTG), l'élvitégravir (EVG) et le raltégravir (RAL), ainsi que sur les inhibiteurs de protéase (IP) et les inhibiteurs non nucléosidiques de la transcriptase inverse (INNTI). L'objectif principal était d'évaluer si le gain de poids variait selon les classes de médicaments et les molécules spécifiques au sein des INSTI. L'échantillon comprenait 1 152 patients adultes naïfs de traitement, recrutés entre janvier 2007 et juin 2016 dans une clinique spécialisée du Tennessee, et suivis pendant 18 mois. Les données, collectées de manière rétrospective, incluaient des mesures de poids, de charge virale et de compte de cellules CD4. Les auteurs ont utilisé des modèles linéaires à effets mixtes multivariés pour prédire les changements de poids, en ajustant pour des facteurs tels que l'âge, le sexe, l'origine ethnique, l'année d'initiation du TAR, le poids initial, la charge virale et le compte de CD4. Les principaux résultats ont montré que les patients sous DTG présentaient le gain de poids le plus important, avec une augmentation moyenne de 6,0 kg à 18 mois, comparé à 2,6 kg pour les INNTI et 0,5 kg pour l'EVG (différences significatives, *p* < 0,05). Le RAL était associé à un gain intermédiaire (3,4 kg), tandis que les IP se situaient à 4,1 kg. Aucune différence significative n'a été observée entre les sexes ou les origines ethniques. Les taux de suppression virologique étaient similaires entre les classes de TAR à 18 mois, bien que les INSTI aient montré une suppression plus rapide dans les premiers mois. Les discussions ont souligné que le gain de poids accru sous DTG pourrait être lié à des mécanismes spécifiques, tels qu'un effet sur les hormones régulant l'appétit, bien que les causes exactes restent à élucider. Les auteurs ont également noté des limites, notamment le design rétrospectif, la prédominance masculine de l'échantillon et l'absence de données sur l'activité physique ou l'apport calorique. Ils concluent que le DTG est associé au gain de poids le plus marqué parmi les INSTI, tout en soulignant la nécessité d'études supplémentaires pour comprendre les implications cliniques à long terme, notamment sur les risques métaboliques et cardiovasculaires.
%
%**Référence :**  
%Bourgi, K., Rebeiro, P. F., Turner, M., Castilho, J. L., Hulgan, T., Raffanti, S. P., Koethe, J. R., & Sterling, T. R. (2020). Greater Weight Gain in Treatment-naive Persons Starting Dolutegravir-based Antiretroviral Therapy. *Clinical Infectious Diseases, 70*(7), 1267–1274. https://doi.org/10.1093/cid/ciz407



%               Erlandson et al. (2021)

%Cette étude, publiée dans *Clinical Infectious Diseases*, visait à évaluer les facteurs associés à la prise de poids chez les personnes vivant avec le VIH (PVVIH) sous suppression virale après un changement de traitement antirétroviral (TAR). L'analyse regroupait les données de 12 essais cliniques randomisés sponsorisés par Gilead Sciences, incluant 7 316 participants (4 166 ayant changé de TAR et 3 150 restés sous leur régime de base stable). Les participants étaient majoritairement des hommes (81,5 %), d'âge médian 46 ans, avec une diversité raciale (22 % de Noirs, 69 % de Blancs) et géographique. Les régimes évalués incluaient des inhibiteurs de l'intégrase (INSTI) comme le dolutégravir (DTG) et le bicitégravir (BIC), ainsi que des analogues nucléosidiques (TDF, TAF, ABC). Les données ont été analysées à l'aide de modèles linéaires à effets mixtes pour les changements de poids et de régression logistique pour les facteurs de risque, avec des ajustements pour l'âge, l'IMC, et les composants du TAR.  
%
%Les résultats ont montré une prise de poids modérée mais significative chez les participants ayant changé de TAR (médiane : 1,6 kg à 48 semaines) comparé à ceux restés sous régime stable (0,4 kg). Les gains les plus importants étaient associés au remplacement du TDF par le TAF (+1,6 kg) ou de l'éfavirenz (EFV) par le rilpivirine (RPV) ou l'elvitégravir/cobicistat (EVG/c). Les facteurs prédictifs incluaient un IMC initial bas (sous-poids/normal) et un jeune âge (≤35 ans), tandis que le sexe, l'origine ethnique et le taux de CD4 n'étaient pas significativement liés. À 48 semaines, 6,4 % des participants ayant changé de TAR ont pris ≥10 % de poids, contre 2,2 % dans le groupe témoin. Les profils lipidiques et la pression artérielle n'ont pas montré de détérioration clinique significative malgré la prise de poids.  
%
%Les discussions soulignent que la prise de poids post-changement pourrait refléter à la fois l'arrêt d'effets suppressifs de poids (par exemple, du TDF ou de l'EFV) et des propriétés favorisant la prise de masse des nouveaux agents comme le TAF ou les INSTI. Les mécanismes sous-jacents restent mal compris, bien que des études pharmacogénomiques suggèrent un rôle du métabolisme de l'EFV. Les limites incluent l'absence de données sur l'alimentation ou l'exercice physique, et la généralisation limitée par la population majoritairement masculine et en bonne santé des essais cliniques. Les auteurs recommandent une surveillance attentive du poids et des conseils sur le mode de vie lors des changements de TAR, tout en appelant à des recherches supplémentaires sur les implications cliniques à long terme.  
%
%**Référence :**  
%Erlandson, K. M., Carter, C. C., Melbourne, K., et al. (2021). Weight Change Following Antiretroviral Therapy Switch in People With Viral Suppression: Pooled Data from Randomized Clinical Trials. *Clinical Infectious Diseases*, *73*(7), e2273–e2284. https://doi.org/10.1093/cid/ciab444


%                   Esber et al. (2022)

%L'étude intitulée *"Weight gain during the dolutegravir transition in the African Cohort Study"* (Esber et al., 2022) vise à évaluer les changements de poids et d'indice de masse corporelle (IMC) chez les personnes vivant avec le VIH (PVVIH) après leur transition vers un régime antirétroviral (ARV) à base de dolutégravir (DTG), spécifiquement le ténofovir/lamivudine/dolutégravir (TLD), dans quatre pays africains (Kenya, Ouganda, Tanzanie, Nigeria). L'objectif principal était d'identifier les risques potentiels de prise de poids associés à ce traitement, particulièrement dans les pays à revenu faible ou intermédiaire, où le DTG est devenu un traitement de première ligne recommandé par l'OMS.  
%
%L'échantillon comprend 2950 PVVIH suivies prospectivement dans le cadre de l'étude AFRICOS, dont 1474 ont transitionné vers le TLD entre 2013 et 2020. Les données démographiques, les régimes ARV, le poids, l'IMC et le rapport taille-hanches ont été collectés tous les 6 mois. Les auteurs ont utilisé des modèles statistiques avancés, incluant des modèles de Cox à risques proportionnels pour estimer les ratios de risque (HR) de développer un IMC ≥ 25 kg/m², ainsi que des modèles mixtes linéaires à effets aléatoires pour analyser les changements moyens de poids, d'IMC et de rapport taille-hanches. Ces modèles ont été ajustés pour tenir compte de facteurs tels que l'âge, le sexe, le site d'étude, la durée du traitement ARV et le nadir des CD4.  
%
%Les principaux résultats montrent que les PVVIH sous TLD avaient un risque accru de 1,77 fois (IC 95 % : 1,22–2,55) de développer un IMC élevé par rapport à celles sous d'autres régimes ARV. En moyenne, les participants sous TLD ont pris 0,68 kg (IC 95 % : 0,32–1,04) de plus que ceux sous d'autres régimes. Parmi ceux ayant transitionné vers le TLD, la prise de poids annuelle moyenne est passée de 0,35 kg/an avant la transition à 1,46 kg/an après (IC 95 % : 1,18–1,75). Les femmes et les participants plus âgés présentaient une prise de poids plus marquée, bien que ces différences ne soient pas toujours significatives après un an de traitement.  
%
%La discussion souligne que ces résultats confirment les observations antérieures sur la prise de poids associée au DTG, même chez des PVVIH déjà sous traitement ARV et bien contrôlées. Les mécanismes potentiels incluent l'inhibition du récepteur MC4R par le DTG, bien que des recherches supplémentaires soient nécessaires pour confirmer cette hypothèse. Les implications cliniques sont préoccupantes, car une prise de poids persistante pourrait augmenter le risque de comorbidités métaboliques (diabète, maladies cardiovasculaires). Les auteurs recommandent une surveillance accrue des patients sous TLD, tout en reconnaissant les avantages supérieurs du DTG en termes d'efficacité et de barrière à la résistance.  
%
%**Référence :**  
%Esber, A. L., Chang, D., Iroezindu, M., Bahemana, E., Kibuuka, H., Owuoth, J., ... & Ake, J. A. (2022). Weight gain during the dolutegravir transition in the African Cohort Study. *Journal of the International AIDS Society*, *25*(2), e25899. https://doi.org/10.1002/jia2.25899


%                Guaraldi et al. (2023),
%
%L'article de Guaraldi et al. (2023), intitulé *"Evidence gaps on weight gain in people living with HIV: a scoping review to define a research agenda"*, publié dans *BMC Infectious Diseases*, vise à identifier les lacunes dans les connaissances sur la prise de poids (WG) chez les personnes vivant avec le VIH (PVVIH) sous thérapie antirétrovirale combinée (cART) et à proposer un agenda de recherche futur. Cette revue scopique suit la méthodologie PRISMA et analyse 175 articles publiés entre 2011 et 2021, issus de bases de données comme PubMed, Embase et WHO Global Index Medicus. Les auteurs se concentrent sur quatre questions de recherche : (1) la définition du WG, (2) sa pathogenèse, (3) l'impact des ART, et (4) sa corrélation avec les outcomes cliniques.  
%
%Les résultats révèlent une absence de consensus sur la définition du WG, avec des critères variés (p. ex., augmentation de 5 % du poids ou de l'IMC). Les études soulignent que le WG est multifactoriel, impliquant des facteurs démographiques (sexe féminin, origine ethnique), liés au VIH (faible taux de CD4, charge virale élevée) et aux ART, notamment les inhibiteurs de l'intégrase (INSTIs comme le dolutégravir) et le ténofovir alafénamide (TAF), associés à une prise de poids significative. Les mécanismes pathogéniques restent mal élucidés, mais incluent des interactions entre l'inflammation résiduelle, le tissu adipeux et les effets métaboliques des ART. Les implications cliniques du WG incluent un risque accru de diabète de type 2, de maladies cardiovasculaires et de syndrome métabolique, bien que le lien direct entre WG et ces comorbidités nécessite des investigations plus poussées.  
%
%La discussion met en lumière les limites des données actuelles, notamment le manque d'études longitudinales et la nécessité de mieux comprendre les mécanismes biologiques. Les auteurs proposent un agenda de recherche pour : (1) standardiser la définition du WG, (2) explorer les interactions entre VIH, ART et métabolisme, (3) clarifier le rôle spécifique des médicaments, et (4) évaluer l'impact indépendant du WG sur les outcomes cliniques. Ils soulignent également l'importance des interventions non pharmacologiques (régime, exercice) et la nécessité d'études incluant des populations diversifiées.  
%
%**Référence :**  
%Guaraldi G, Bonfanti P, Di Biagio A, et al. Evidence gaps on weight gain in people living with HIV: a scoping review to define a research agenda. *BMC Infect Dis.* 2023;23:230. https://doi.org/10.1186/s12879-023-08174-3  
%
%**Discussion :**  
%Les auteurs concluent que malgré l'abondance de littérature sur le WG chez les PVVIH, les lacunes persistent, notamment en raison de l'hétérogénéité des définitions et des méthodologies. Ils appellent à des recherches futures pour élucider les mécanismes sous-jacents, optimiser les stratégies de gestion du poids et évaluer les risques à long terme. Les conflits d'intérêts sont limités, avec un financement indépendant, renforçant la crédibilité des conclusions. Cette revue sert de base pour orienter les efforts de recherche vers une prise en charge plus personnalisée des PVVIH face aux défis métaboliques liés aux ART modernes.  
%  

%           Huis in 't Veld et al. (2015)

%L'étude intitulée *"Determinants of weight evolution among HIV-positive patients initiating antiretroviral treatment in low resource settings"* vise à analyser l'évolution du poids corporel chez les patients séropositifs sous traitement antirétroviral (ART) dans des régions à ressources limitées, où les paramètres cliniques, tels que les variations de poids, sont souvent utilisés pour évaluer la réponse au traitement. L'objectif principal était d'identifier les facteurs associés aux changements de poids au cours des deux premières années de traitement. Les données ont été extraites des bases de données prospectives de l'*International Epidemiologic Databases to Evaluate AIDS (IeDEA)*, un réseau mondial collectant des informations cliniques de routine. L'échantillon final comprenait 205 571 patients adultes issus de cinq régions (Afrique australe, Afrique de l'Est, Afrique de l'Ouest, Afrique centrale et Asie-Pacifique), sélectionnés sur la base de critères incluant des données de poids, de genre, de régime ART et de compte de CD4 à l'initiation du traitement. Les femmes représentaient 61,4 % des participants, et la majorité des patients provenaient d'Afrique australe (67,7 %). Les modèles statistiques utilisés incluaient un modèle linéaire mixte (LMM) pour analyser l'évolution du poids sur 24 mois et une régression logistique pour évaluer le risque de perte de poids supérieure à 5 % lors de la deuxième année. Les résultats ont montré que le gain de poids était plus marqué durant la première année, particulièrement chez les patients en mauvais état clinique initial (faible compte de CD4, stade avancé de l'infection, faible taux d'hémoglobine). Les régimes à base de ténofovir (TDF) et d'éfavirenz (EFV) étaient associés à un gain de poids plus important, tandis que la stavudine (D4T) était liée à une prise de poids moindre, voire à une perte de poids lors de la deuxième année, probablement en raison de la lipoatrophie, un effet secondaire connu. Les différences géographiques étaient également notables, avec des gains de poids plus élevés en Afrique centrale et de l'Ouest. Les discussions soulignent l'importance du gain de poids comme indicateur de succès thérapeutique, tout en mettant en garde contre les risques d'obésité dans certaines populations. Les limites de l'étude incluent le biais potentiel lié à la représentativité des centres participants et l'absence de données sur l'adhésion au traitement ou les co-infections. En conclusion, l'étude confirme que le poids est un marqueur clinique utile pour le suivi des patients sous ART, tout en recommandant une vigilance accrue concernant les effets secondaires des médicaments, notamment la D4T, dont l'utilisation devrait être limitée selon les directives de l'OMS.

%**Référence :**  
%Huis in 't Veld, D., Balestre, E., Buyze, J., et al. (2015). *Determinants of weight evolution among HIV-positive patients initiating antiretroviral treatment in low resource settings*. Journal of Acquired Immune Deficiency Syndromes, 70(2), 146–154. doi:10.1097/QAI.0000000000000691.


%                   Kim et al. (2022)

%Cette étude rétrospective a exploré les effets des régimes antirétroviraux à base d’inhibiteurs de transfert de brins d’intégrase (INSTI) sur la prise de poids et les modifications du profil lipidique chez des personnes coréennes vivant avec le VIH (PLWH). L'objectif principal était d'évaluer la relation entre ces traitements et la prise de poids sur une période de 24 mois, ainsi que d'identifier les facteurs de risque associés. L'étude a inclus 469 participants (179 naïfs de traitement et 290 expérimentés) recrutés dans trois hôpitaux universitaires sud-coréens entre mai 2014 et décembre 2020. Les participants ont reçu l'un des trois régimes à base d'INSTI : TDF/F/EVG/c (ténofovir disoproxil fumarate/emtricitabine/elvitégravir/cobicistat), TAF/F/EVG/c (ténofovir alafénamide/emtricitabine/elvitégravir/cobicistat), ou ABC/3TC/DTG (abacavir/lamivudine/dolutégravir). Les données ont été analysées à l'aide de tests statistiques tels que l'ANOVA, les tests de Tukey, et des modèles de régression logistique pour identifier les facteurs de risque.  
%
%Les résultats ont montré une prise de poids significative dans les deux groupes, avec des gains moyens plus élevés pour les régimes TAF/F/EVG/c et ABC/3TC/DTG par rapport au TDF/F/EVG/c. Dans le groupe naïf de traitement, le gain moyen était de 5,5 kg pour TAF/F/EVG/c, 4,9 kg pour ABC/3TC/DTG, et 2,5 kg pour TDF/F/EVG/c. Des changements dans les catégories d'IMC (augmentation du surpoids et de l'obésité) ont été observés, particulièrement avec TAF/F/EVG/c et ABC/3TC/DTG. Le profil lipidique a également été affecté, avec une augmentation significative du cholestérol total (TC), du LDL-C et des triglycérides (TG) pour TAF/F/EVG/c. Les facteurs de risque identifiés comprenaient un faible taux de CD4+ (<100 cellules/mm³), une charge virale élevée (≥100 000 copies/mL), l'absence d'exercice physique, et le régime TAF/F/EVG/c dans le groupe naïf, tandis que l'âge <45 ans, un IMC <25 kg/m², et l'absence d'exercice étaient des facteurs dans le groupe expérimenté.  
%
%Les discussions soulignent que la prise de poids sous INSTI est un phénomène global, mais avec des variations raciales, et confirment l'impact défavorable de TAF sur les lipides. L'étude met en évidence l'importance de l'exercice physique comme facteur protecteur contre la prise de poids. Les limites incluent le caractère rétrospectif, l'absence de données complètes pour tous les participants, et l'omission de facteurs comme le régime alimentaire. Les auteurs recommandent une surveillance étroite du poids et des paramètres métaboliques lors de la prescription d'INSTI, en particulier chez les patients présentant des facteurs de risque, et encouragent les modifications du mode de vie pour atténuer ces effets.  
%
%**Référence :**  
%Kim, J., Nam, H.-J., Jung, Y.-J., Lee, H.-J., Kim, S.-E., Kang, S.-J., Park, K.-H., Chang, H.-H., Kim, S.-W., Chung, E.-K., Kim, U. J., & Jung, S. I. (2022). Weight Gain and Lipid Profile Changes in Koreans with Human Immunodeficiency Virus undergoing Integrase Strand Transfer Inhibitor-Based Regimens. *Infection & Chemotherapy*, *54*(3), 419–432. https://doi.org/10.3947/jc.2022.0063



%

%                 Patel et Malvestutto (2024)

%L'article de Patel et Malvestutto (2024), intitulé *"Beyond the Numbers: Weight Gain Risk Factors, Implications, and Interventions among Individuals with HIV"*, publié dans le *Journal of AIDS and HIV Treatment*, explore les facteurs de risque, les implications et les stratégies de gestion de la prise de poids chez les personnes vivant avec le VIH (PVVIH). L'objectif principal de cette revue narrative est de synthétiser les données existantes sur la prise de poids liée à la thérapie antirétrovirale (ART), en mettant en lumière les mécanismes multifactoriels impliqués et en proposant des interventions personnalisées. Les auteurs analysent un large éventail d'études, incluant des essais cliniques randomisés (comme ADVANCE et NAMSAL), des méta-analyses, des études observationnelles et des données de cohortes, pour évaluer l'impact des différents schémas thérapeutiques (notamment les inhibiteurs de l'intégrase comme le dolutégravir (DTG) et le ténofovir alafénamide (TAF)) sur la prise de poids.  
%
%Les principaux résultats révèlent que la prise de poids chez les PVVIH est influencée par des facteurs démographiques (sexe féminin, race noire), liés au VIH (faible taux de lymphocytes CD4, charge virale élevée) et aux ART (régimes à base d'INSTI et TAF). Les études montrent que le DTG et le TAF sont associés à une prise de poids significative, tandis que des agents plus anciens comme l'efavirenz (EFV) et le ténofovir disoproxil fumarate (TDF) ont des effets suppressifs sur le poids. Les essais de substitution (par exemple, le passage du TAF au TDF) ont démontré une perte de poids chez certaines populations, en particulier les femmes. Les implications cliniques incluent un risque accru de comorbidités cardiométaboliques (diabète, maladies cardiovasculaires) et de cancers non liés au SIDA.  
%
%Pour la gestion de la prise de poids, les auteurs proposent une approche individualisée combinant : (1) l'adaptation des schémas ART (éviter les combinaisons à haut risque comme DTG + TAF), (2) des modifications du mode de vie (régime hypocalorique, activité physique), et (3) des interventions pharmacologiques (agonistes des récepteurs du GLP-1 comme le sémaglutide). Cependant, les mécanismes sous-jacents restent mal compris, et des recherches supplémentaires sont nécessaires pour clarifier le rôle causal des ART et optimiser les stratégies thérapeutiques.  
%
%**Référence :**  
%Patel YS, Malvestutto CD. Beyond the Numbers: Weight Gain Risk Factors, Implications, and Interventions among Individuals with HIV. *J AIDS HIV Treat.* 2024;6(1):1-10. doi: [insérer DOI si disponible].  
%
%**Discussion :**  
%Les auteurs soulignent les limites des données actuelles, notamment le manque d'études longitudinales sur les effets à long terme de la prise de poids et les interactions complexes entre les ART et le métabolisme. Ils appellent à des recherches futures pour explorer les mécanismes physiologiques, évaluer l'efficacité des interventions préventives et optimiser la prise en charge des comorbidités liées à l'obésité chez les PVVIH vieillissantes. Les conflits d'intérêts sont minimes, avec un financement indépendant, renforçant la crédibilité des conclusions.  
%  
%  

%                  Pedersen et al., 2023

%L'essai randomisé AVERTAS (Pedersen et al., 2023) a évalué l'impact du passage d'un traitement antirétroviral triple (dolutégravir/abacavir/lamivudine, DTG/ABC/3TC) à un régime double (dolutégravir/lamivudine, DTG/3TC) sur le poids, la composition corporelle et les paramètres métaboliques chez 70 personnes vivant avec le VIH (PVVIH) virologiquement contrôlées au Danemark. Les participants ont été randomisés en deux groupes (2:1) pour soit poursuivre le traitement triple (groupe témoin), soit passer au double (intervention), avec un suivi de 48 semaines. Les données ont été recueillies via des examens cliniques, des analyses sanguines, des scanners abdominaux, des DEXA et des IRM cardiaques (sous-étude). Les résultats ont montré une réduction significative du poids dans le groupe DTG/3TC (−1,0 kg) comparé au groupe témoin (+2,0 kg), confirmant l'effet pondérogène de l'abacavir. Aucune différence significative n'a été observée sur les paramètres métaboliques (lipides, glycémie) ou cardiovasculaires (score calcique, épaisseur intima-média), suggérant que le retrait de l'abacavir améliore le profil pondéral sans altérer le contrôle métabolique. Ces résultats soutiennent l'utilisation des bithérapies comme alternative pour limiter la prise de poids chez les PVVIH sous INSTI, tout en maintenant l'efficacité virologique. L'étude souligne l'importance d'individualiser les traitements selon les risques cardiométaboliques.


%              Sax et al. (2020)

%L'étude intitulée *"Weight Gain Following Initiation of Antiretroviral Therapy: Risk Factors in Randomized Comparative Clinical Trials"* vise à identifier les facteurs associés à la prise de poids chez les personnes vivant avec le VIH (PVVIH) initiant un traitement antirétroviral (TAR), en mettant l'accent sur les différences entre les schémas thérapeutiques et les caractéristiques démographiques. L'objectif principal était d'évaluer l'impact des médicaments antirétroviraux, des facteurs liés au VIH (tels que la charge virale et le taux de CD4) et des caractéristiques des patients (sexe, race, âge) sur l'évolution pondérale. Les données proviennent d'une analyse regroupée de huit essais cliniques randomisés et contrôlés, sponsorisés par Gilead Sciences, menés entre 2003 et 2015, incluant 5 680 participants naïfs de traitement avec un suivi moyen de deux ans. Les méthodes statistiques comprenaient des modèles linéaires mixtes pour analyser les changements de poids longitudinaux et des régressions logistiques pour évaluer les risques de prise de poids significative (≥10 %). Les principaux résultats ont révélé que la prise de poids était ubiquitaire mais variable selon les schémas thérapeutiques : les inhibiteurs de l'intégrase (INSTIs), notamment le dolutégravir (DTG) et le bicitégravir (BIC), étaient associés à une prise de poids plus importante que les inhibiteurs de la protéase (IP) ou les inhibiteurs non nucléosidiques de la transcriptase inverse (INNTI), avec une différence moyenne de 1,3 à 2,5 kg après 96 semaines. Parmi les INNTI, la rilpivirine (RPV) entraînait une prise de poids plus marquée que l'éfavirenz (EFV). Concernant les inhibiteurs nucléosidiques de la transcriptase inverse (INTI), le ténofovir alafénamide (TAF) était associé à une augmentation pondérale plus importante que le ténofovir disoproxil fumarate (TDF) ou l'abacavir (ABC). Les facteurs démographiques jouaient également un rôle significatif : les femmes, les personnes noires (en particulier les femmes noires) et les patients avec un faible taux de CD4 ou une charge virale élevée au départ présentaient une prise de poids plus importante. Les discussions soulignent que cette prise de poids pourrait résulter d'une combinaison de facteurs, incluant un "retour à la santé" chez les patients immunodéprimés, une meilleure tolérance des nouveaux schémas thérapeutiques (réduisant les effets gastro-intestinaux) et des mécanismes pharmacologiques spécifiques (comme l'interaction potentielle du DTG avec le récepteur MC4R impliqué dans la régulation de l'appétit). Cependant, les implications métaboliques à long terme (diabète, maladies cardiovasculaires) restent incertaines en raison de la durée limitée des essais. Les auteurs concluent que la prise de poids sous TAR est multifactorielle et nécessite une surveillance clinique accrue, en particulier chez les populations à risque (femmes noires, patients immunodéprimés), et appellent à des recherches supplémentaires pour élucider les mécanismes sous-jacents et optimiser la gestion pondérale.  

%**Référence :**  
%Sax, P. E., Erlandson, K. M., Lake, J. E., et al. (2020). *Weight Gain Following Initiation of Antiretroviral Therapy: Risk Factors in Randomized Comparative Clinical Trials*. Clinical Infectious Diseases, 71(6), 1379–1389. https://doi.org/10.1093/cid/ciz999


%                 Taramasco et al., 2020

%%Cette étude prospective multicentrique italienne (Taramasco et al., 2020) visait à évaluer les facteurs associés à la prise de poids chez les personnes vivant avec le VIH (PVVIH) sous traitement à base de dolutégravir (DTG), en analysant notamment l'impact des différents schémas antirétroviraux combinés. Menée dans le cadre de la cohorte SCOLTA, elle a inclus 713 participants (25,3 % de femmes, 91 % de Caucasiens) naïfs (27,4 %) ou expérimentés (72,6 %) traités par DTG associé à divers schémas (ABC/3TC, TDF/FTC, TAF/FTC, etc.) entre 2014 et 2019. Les données de poids, de paramètres métaboliques et immunologiques ont été collectées prospectivement pendant 24 mois, avec des analyses statistiques incluant des modèles linéaires ajustés et une régression de Cox pour identifier les facteurs de risque. Les résultats ont montré une prise de poids significative à 6 et 12 mois (+1,2 kg et +2,4 kg chez les naïfs), particulièrement marquée avec les combinaisons TAF/FTC+DTG (risque multiplié par 3,8) et TDF/FTC+DTG (risque multiplié par 1,9). Les facteurs indépendamment associés à une prise de poids >10 % incluaient un statut naïf (HR=2,24), un taux de CD4 <200 cellules/mm³ (HR=1,84), et l'éradication du VHC (HR=1,84), tandis qu'un poids initial élevé et le sexe féminin étaient protecteurs. Aucun impact significatif sur les lipides ou la glycémie n'a été observé, et l'incidence du syndrome métabolique est restée faible (9,9 cas/1000 personnes-années). Ces résultats suggèrent que la prise de poids sous DTG est influencée par le statut immunovirologique initial et le choix du schéma associé, avec un rôle potentiel du TAF, sans conséquence métabolique majeure à court terme.


%            Valenzuela-Rodriguez et al., 2023

%L'étude *"Weight and Metabolic Outcomes in Naive HIV Patients Treated with Integrase Inhibitor-Based Antiretroviral Therapy: A Systematic Review and Meta-Analysis"* (Valenzuela-Rodriguez et al., 2023) vise à évaluer l'impact des inhibiteurs de l'intégrase (INSTI) sur le poids et les paramètres métaboliques chez des patients naïfs de traitement vivant avec le VIH, en les comparant aux autres classes d'antirétroviraux (inhibiteurs de protéase et NNRTI). Les auteurs ont systématiquement analysé six essais cliniques randomisés (totalisant 3 521 patients), publiés entre 2009 et 2020, en suivant les directives PRISMA. Les études incluses comparaient différents INSTI (raltégravir, elvitégravir, dolutégravir) à des schémas à base d'éfavirenz, d'atazanavir/ritonavir ou de darunavir/ritonavir, avec des périodes de suivi variant de 48 à 96 semaines. Pour l'analyse statistique, un modèle à effets aléatoires a été privilégié afin de tenir compte de l'hétérogénéité attendue entre les études, en utilisant la méthode Paule-Mandel pour estimer la variance entre les études (τ²) et la méthode Hartung-Knapp pour ajuster les intervalles de confiance, garantissant ainsi des résultats plus robustes malgré le nombre limité d'études. Les principaux résultats révèlent que les INSTI sont associés à une augmentation moyenne du poids de 2,15 kg (IC 95 % : 1,40–2,90) par rapport aux autres classes d'antirétroviraux, avec une hétérogénéité nulle (I² = 0 %), indiquant une cohérence entre les études pour cet outcome. En revanche, les INSTI montrent des effets métaboliques favorables, avec des réductions significatives du cholestérol total (–13,44 mg/dL), du LDL (–11,37 mg/dL) et des triglycérides (–20,70 mg/dL), bien que ces résultats présentent une hétérogénéité élevée (I² > 90 %), reflétant des variations importantes entre les études. Les sous-analyses suggèrent que le dolutégravir est particulièrement associé à une prise de poids plus marquée, tandis que les schémas contenant du ténofovir alafénamide (TAF) exacerbent cet effet. Les auteurs discutent des mécanismes potentiels, comme l'inhibition des récepteurs mélanocortine-4 (MC4R) par le dolutégravir, et soulignent les implications cliniques, notamment le risque accru de complications métaboliques à long terme (diabète, maladies cardiovasculaires) malgré le bénéfice lipidique. Les limites incluent l'hétérogénéité non résolue pour les marqueurs lipidiques, la durée de suivi relativement courte et les biais potentiels dans certains essais (ex. : absence de masquage). En conclusion, bien que les INSTI offrent un profil métabolique avantageux, leur association avec une prise de poids modeste nécessite une surveillance accrue, particulièrement chez les populations vulnérables (femmes, patients africains/hispaniques). Les auteurs appellent à des études complémentaires pour évaluer les effets à long terme et les différences entre les molécules spécifiques au sein de la classe des INSTI. Cette méta-analyse, bien que rigoureuse, met en lumière la complexité des compromis entre efficacité, tolérance et sécurité métabolique dans le traitement du VIH.

%L'étude "Weight and Metabolic Outcomes in Naive HIV Patients Treated with Integrase Inhibitor-Based Antiretroviral Therapy: A Systematic Review and Meta-Analysis" (Valenzuela-Rodriguez et al., 2023) évalue l'impact des inhibiteurs de l'intégrase (INSTI) sur le poids et les paramètres métaboliques chez des patients naïfs de traitement vivant avec le VIH, comparativement aux inhibiteurs de protéase et NNRTI. Les auteurs ont analysé six essais cliniques randomisés clés (totalisant 3 521 patients) publiés entre 2009 et 2020 :

%Lennox (2009) : Comparaison raltégravir vs éfavirenz (n=563, 48 semaines) révélant un meilleur profil lipidique sous INSTI.
%
%Rockstroh (2013) : Elvitégravir vs atazanavir/ritonavir (n=708, 96 semaines), premier essai contre un PI boosté.
%
%Calmy (2020 - NAMSAL) : Dolutégravir vs éfavirenz (n=613, 96 semaines) montrant +2.3 kg sous INSTI dans une population majoritairement féminine.
%
%Venter (2020 - ADVANCE) : Trois bras comparant dolutégravir/TAF/FTC, dolutégravir/TDF/FTC et éfavirenz/TDF/FTC (n=1 053), démontrant l'impact du TAF sur la prise de poids.
%
%Walmsley (2013 - SINGLE) : Dolutégravir/ABC/3TC vs éfavirenz/TDF/FTC (n=833) avec +2.1 kg sous dolutégravir.
%
%Clotet (2014 - FLAMINGO) : Comparaison directe dolutégravir vs darunavir/ritonavir (n=484), confirmant la supériorité métabolique des INSTI.
%
%Ces essais multicentriques (Afrique, Europe, Amériques, Asie), bien que méthodologiquement hétérogènes (durée 48-96 semaines, populations variées), ont été analysés via un modèle à effets aléatoires (méthodes Paule-Mandel pour τ² et Hartung-Knapp pour les IC) pour intégrer leur diversité. Les résultats montrent une augmentation moyenne de poids de 2,15 kg (IC 95% : 1,40-2,90) sous INSTI (I²=0%), particulièrement marquée avec le dolutégravir et les schémas contenant du TAF, contrebalancée par des améliorations lipidiques (cholestérol total : -13,44 mg/dL ; LDL : -11,37 mg/dL ; triglycérides : -20,70 mg/dL), bien qu'avec une hétérogénéité élevée (I²>90%).
%
%Les auteurs discutent les mécanismes potentiels (ex. : inhibition des récepteurs MC4R par le dolutégravir) et les implications cliniques, notamment le risque de complications métaboliques à long terme malgré le bénéfice lipidique. Les limites incluent l'hétérogénéité résiduelle pour les marqueurs lipidiques, la durée de suivi limitée à 2 ans maximum, et des biais méthodologiques dans certains essais (ex. : absence de masquage). La conclusion souligne la nécessité de surveiller la prise de poids sous INSTI, surtout chez les populations vulnérables (femmes, patients africains/hispaniques), et appelle à des études complémentaires sur les effets à long terme et les différences moléculaires au sein de la classe des INSTI. Cette méta-analyse rigoureuse illustre ainsi le compromis complexe entre efficacité antivirale, tolérance et sécurité métabolique dans la prise en charge du VIH.

%                      Vizcarra et al. (2020)

%L’étude menée par Vizcarra et al. (2020) vise à évaluer de manière précise les effets d’un passage à une bithérapie à base de dolutégravir/rilpivirine sur la composition corporelle, notamment en termes de gain pondéral et de répartition de la masse grasse, chez des patients vivant avec le VIH, virologiquement supprimés et lourdement prétraités. Cette sous-analyse de l’étude prospective DOLBi a inclus 54 patients (37 sous dolutégravir/rilpivirine et 17 dans le groupe témoin sous darunavir/lamivudine) sélectionnés parmi 138 participants initiaux selon des critères stricts d’inclusion (suppression virologique, absence de co-infection VHB, de lipodystrophie, de grossesse, ou d’exposition antérieure aux inhibiteurs d'intégrase). La méthodologie repose sur un suivi de 12 mois avec mesure du poids et de la composition corporelle par absorptiométrie biénergétique à rayons X (DXA) réalisée avant et après le switch thérapeutique. Les analyses statistiques ont mobilisé des tests de Wilcoxon et Mann–Whitney pour les comparaisons, ainsi qu’un modèle de régression linéaire pour identifier les facteurs associés à l’augmentation de la masse grasse. Les résultats montrent une augmentation significative du poids (1,8 kg ; +2,5 % ; *p*=0,03) et de l’IMC (+0,6 kg/m² ; *p*=0,02) dans le groupe dolutégravir/rilpivirine après 12 mois, tandis que le groupe darunavir/lamivudine présente un gain pondéral moindre et non significatif. Toutefois, les deux groupes affichent des augmentations similaires de la masse grasse dans le tronc, les bras et les jambes, sans changement significatif de la masse maigre ni de la répartition graisseuse (le rapport masse grasse tronc/jambes reste stable), ce qui suggère que le gain de poids n’est pas de nature lipodystrophique. L’analyse multivariée révèle que l’augmentation de la masse grasse est inversement associée à la masse grasse initiale et au taux de CD4, et positivement corrélée au délai entre les deux DXA. Ces résultats renforcent l’idée que les patients avec un faible stock initial de tissu adipeux ou un statut immunitaire plus altéré présentent une tendance accrue à l’accumulation de graisse après switch. La discussion souligne que bien que le gain soit modeste, ces modifications doivent être considérées dans le contexte du risque cardiovasculaire et métabolique, surtout chez des patients déjà âgés et longuement traités. Les auteurs notent cependant plusieurs limites, dont la taille restreinte de l’échantillon, l’absence de randomisation du schéma thérapeutique, l’impossibilité d’évaluer la graisse viscérale de manière directe, ainsi que le risque de facteurs de confusion non mesurés. En conclusion, ce travail met en évidence que le passage à une bithérapie à base de dolutégravir induit une prise de poids modérée mais significative, principalement liée à une augmentation de la masse grasse, sans altération de la masse musculaire ni modification marquée du profil lipodystrophique, ce qui souligne la nécessité d’un suivi métabolique adapté dans ce contexte.

%**Référence :**  
%Vizcarra P, Vivancos MJ, Pérez-Elías MJ, Moreno A, Casado JL. (2020). *Weight gain in people living with HIV switched to dual therapy: changes in body fat mass*. AIDS, 34(1), 155–157. https://doi.org/10.1097/QAD.0000000000002421.


%                Wohl et al. (2024)
%
%L’article de Wohl et al. (2024) propose une revue critique des données cliniques concernant les effets des antirétroviraux sur la prise de poids chez les personnes vivant avec le VIH. En s’appuyant sur une synthèse d’essais cliniques randomisés majeurs (ADVANCE, DISCOVER, GEMINI, TANGO, SALSA), les auteurs analysent les variations pondérales observées selon les régimes thérapeutiques, en particulier ceux contenant des inhibiteurs de l’intégrase (DTG, BIC) et le ténofovir alafénamide (TAF). Contrairement aux inquiétudes soulevées dans certaines études observationnelles, ils montrent que ces molécules ont un effet neutre sur le poids, et que la perte ou la limitation de prise de poids observée dans d’autres groupes est probablement liée à l’effet atténuateur du TDF ou de l’efavirenz. La revue souligne l’importance de considérer d’autres facteurs explicatifs (sexe, génétique, retour à la santé) et appelle à la prudence dans les changements de traitement motivés uniquement par la prise de poids.
%






\section{Revue de littérature} 

L’objectif ici est de faire un état des lieux des connaissances empiriques disponibles sur la problématique de la prise de poids chez les patients vivant avec le VIH sous traitement antirétroviral (ARV)



\subsection{Impact des classes d’antirétroviraux sur la prise de poids}

La prise de poids chez les personnes vivant avec le VIH (PVVIH) sous traitement antirétroviral (TAR) est devenue une préoccupation majeure depuis l’introduction des schémas thérapeutiques modernes. En particulier, les inhibiteurs de l’intégrase (INI) et le ténofovir alafénamide (TAF) ont été associés à des modifications pondérales notables. Plusieurs études récentes ont permis d’identifier une association significative entre certaines classes de molécules antirétrovirales et une prise de poids, indépendamment de la suppression virale.\\


\begin{itemize}
	\item[\ding{118}]	\textbf{ \large Inhibiteurs de l'intégrase (INI)}
\end{itemize} 


Les inhibiteurs de l’intégrase, et tout particulièrement le dolutégravir (DTG), sont régulièrement associés à une prise de poids plus marquée chez les patients traités. \vspace{0.3cm} 

L’étude rétrospective de cohorte menée par \textcite{bourgi2020greater} a analysé l’impact des différentes classes d’antirétroviraux sur la prise de poids chez 1 152 patients n’ayant jamais reçu de traitement, aux États-Unis. Elle a comparé la variation de poids sur 18 mois entre trois types de régimes : ceux à base d’inhibiteurs de l’intégrase (INI : dolutégravir, elvitrégravir, raltégravir), d’inhibiteurs de la protéase (IP) et d’inhibiteurs non nucléosidiques de la transcriptase inverse (INNTI). Grâce à des modèles linéaires à effets mixtes, les auteurs ont observé que le dolutégravir entraînait un gain moyen de 6,0 kg après 18 mois, soit plus du double des 2,6 kg constatés sous INNTI. Le raltégravir était associé à une prise moyenne de 3,4 kg, nettement plus élevée que celle observée avec l’elvitrégravir (0,5 kg). Ces résultats confirment l’hypothèse d’un effet significatif sur le poids des inhibiteurs de l’intégrase, en particulier du dolutégravir (DTG).\vspace{0.3cm} 

Ces résultats sont confirmés par \textcite{taramasso2020factors}, qui s’inscrivent dans le débat sur la prise de poids excessive chez les personnes vivant avec le VIH (PVVIH) sous traitements à base de dolutégravir (DTG). Bien que le DTG soit largement utilisé en première ligne pour son efficacité et sa bonne tolérance, des cas de prise de poids importante ont été signalés. Pour mieux comprendre ce phénomène, les auteurs ont analysé des données de la cohorte italienne SCOLTA (Surveillance Cohort Long-Term Toxicity Antiretrovirals), incluant 713 adultes, naïfs ou expérimentés, ayant commencé un traitement à base de DTG entre 2014 et 2019. Les participants ont été suivis sur une médiane de 28 mois. En utilisant un modèle de Cox pour évaluer le risque de prise de poids supérieure à 10\%, ils ont constaté que les combinaisons TAF/FTC+DTG et TDF/FTC+DTG étaient associées à un risque plus élevé, particulièrement pour TAF/FTC+DTG (HR=3,80) et, dans une moindre mesure, pour TDF/FTC+DTG (HR=1,92), comparativement à DTG+ABC/3TC. Ces résultats montrent que, si le DTG influence significativement le poids, la combinaison avec d’autres molécules, en particulier TAF/FTC, joue un rôle déterminant dans l’ampleur de cette prise de poids.\vspace{0.3cm} 


En Afrique, \textcite{calmy2020dolutegravir}, à travers l’essai NAMSAL mené au Cameroun entre 2016 et 2021, ont étudié l’évolution du poids et des paramètres métaboliques chez des patients sous deux traitements : DTG + TDF/3TC et EFV 400 mg + TDF/3TC, sur une période de 48 semaines. Les résultats ont montré une prise de poids médiane de +5 kg avec le DTG, contre +3 kg avec l’EFV. De même, l’essai ADVANCE (2017-2018) \parencite{Venter2020}, mené en Afrique du Sud, a comparé les schémas DTG + TAF + FTC, DTG + TDF + FTC et EFV + TDF + FTC. Après 48 semaines, la prise de poids (incluant la masse maigre et la masse grasse) était plus importante dans le groupe TAF : +7,1 kg dans le groupe DTG + TAF + FTC, +4,3 kg dans le groupe DTG + TDF + FTC et +2,3 kg dans le groupe de soins standard. Ces résultats confirment que l’association du DTG avec le TAF favorise une prise de poids plus importante.\vspace{0.3cm} 


Toujours dans le contexte africain, \textcite{esber2022weight} ont étudié l’effet du passage au traitement Ténofovir/Lamivudine/Dolutégravir (TLD) sur l’IMC des PVVIH dans quatre pays : Kenya, Ouganda, Tanzanie et Nigeria. L’étude a porté sur 2 950 personnes, dont 1 474 sont passées au TLD et ont été suivies pendant 6 mois. Les chercheurs ont utilisé des modèles de Cox pour estimer le risque de surpoids (IMC $\geq$ 25 kg/m²) et des modèles mixtes pour suivre l’évolution du poids. Les résultats montrent que les patients sous TLD avaient un risque 1,77 fois plus élevé d'être en surpoids par rapport à ceux sous d’autres traitements. Et chez ces patients la prise de poids annuelle est passée de 0,35 kg avant le changement à 1,46 kg après, ce qui montre une augmentation nette du gain pondéral sous DTG.\\ 






\begin{itemize}
	\item[\ding{118}]	\textbf{ \large Autres classes : INNTI, INTI}
\end{itemize} 



Si les inhibiteurs de l'intégrase (INI) sont au cœur des préoccupations, d’autres classes d’antirétroviraux ont également un impact sur la prise de poids, bien que souvent moindre ou même inverse.\vspace{0.3cm} 



Les inhibiteurs non nucléosidiques de la transcriptase inverse (INNTI), et en particulier l’efavirenz (EFV), exercent un effet suppressif sur la prise de poids chez les personnes vivant avec le VIH. Autrement dit, les patients traités avec l’EFV tendent à maintenir un poids stable, voire à perdre légèrement du poids. De même, le ténofovir disoproxil fumarate (TDF), un inhibiteur nucléosidique plus ancien, est associé à un effet similaire de limitation du poids. Plusieurs études, dont celles de \textcite{patel2024beyond} et \textcite{erlandson2021weight}, montrent que le remplacement de l’EFV ou du TDF par des molécules plus récentes, comme le ténofovir alafénamide (TAF) ou un inhibiteur de l’intégrase (INI), est souvent suivi d’une prise de poids rapide chez les patients. Inversement, certains essais de substitution, par exemple le passage du TAF au TDF, ont montré une perte de poids chez certaines populations, confirmant l’effet suppressif des anciennes molécules et suggérant que l’efavirenz pourrait agir en inhibant l’appétit ou en modifiant le métabolisme énergétique.\vspace{0.3cm} 



L’impact différentiel des nucléos(t)ides, notamment le remplacement du TDF par le TAF, a été aussi examiné  par \textcite{ando2021}, qui, à travers des modèles à effets mixtes, montrent que les schémas contenant du TAF sont liés aux prises pondérales les plus élevées, tandis que ceux intégrant du TDF ou de l’abacavir (ABC) entraînent des augmentations plus modestes. La revue systématique de \textcite{guaraldi2023evidence}, fondée sur une analyse approfondie des études cliniques, désigne également le TAF comme un contributeur majeur aux effets métaboliques indésirables dans les traitements modernes. Ces résultats suggèrent des mécanismes métaboliques spécifiques liés au TAF, bien que les voies exactes restent encore à élucider.\\ 





En somme, les études montrent que les inhibiteurs de l’intégrase (dolutégravir) et les inhibiteurs nucléosidiques de la transcriptase inverse (TAF) sont les classes d'antirétroviraux les plus souvent associés à une prise de poids chez les personnes vivant avec le VIH. Bien que les mécanismes précis restent partiellement inconnus, cette prise de poids peut augmenter le risque de maladies cardiovasculaires et métaboliques, ce qui souligne l’importance de surveiller régulièrement le poids et d’adapter les traitements selon les besoins individuels. \vspace{0.4cm}

Par ailleurs, la prise de poids n’est pas uniquement déterminée par le type de traitement : des facteurs individuels, contextuels et géographiques peuvent également influencer l’évolution pondérale des patients. 


%Devant la complexité des interactions entre traitements antirétroviraux et prise de poids, plusieurs chercheurs militent pour une individualisation accrue des schémas thérapeutiques. L’essai randomisé AVERTAS \textcite{pedersen2023changes} a utilisé des tests statistiques comparatifs pour démontrer qu’un passage de la trithérapie DTG/ABC/3TC à la bithérapie DTG/3TC permettait de réduire en moyenne le poids de 1,0 kg chez 70 patients danois, suggérant un effet pondérogène de l’abacavir. 

   
\subsection{ Facteurs individuels, contextuels et géographiques influençant la prise de poids}

La prise de poids observée chez les personnes vivant avec le VIH (PVVIH) sous thérapie antirétrovirale (TAR) ne peut être attribuée exclusivement aux effets pharmacologiques des antirétroviraux. En effet, plusieurs études mettent en évidence un ensemble de facteurs individuels, contextuels et géographiques jouant un rôle déterminant dans la variabilité des réponses pondérales.\\




\begin{itemize}
	\item[\ding{118}]	\textbf{ \large Facteurs individuels}
\end{itemize} 


Les caractéristiques propres aux patients vivant avec le VIH (PVVIH) jouent un rôle déterminant dans la variabilité de la prise de poids observée sous traitement antirétroviral (TAR). Les données issues d’études de cohorte et d’essais randomisés montrent que certains profils démographiques et cliniques sont plus à risque. \textcite{sax2020weight}, dans une étude regroupant 5\,680 participants, ont mis en évidence que les femmes, les personnes d’origine africaine ou afro-américaine, ainsi que ceux présentant une charge virale élevée ou un taux de CD4 bas à l’initiation du traitement, prennent significativement plus de poids. De manière concordante, \textcite{patel2024beyond} identifient comme facteurs prédictifs le sexe féminin, l’origine ethnique noire et un indice de masse corporelle (IMC) initialement bas. \textcite{patel2024beyond} évoquent le phénomène de \enquote{return-to-health} pour expliquer pourquoi les patients ayant un indice de masse corporelle (IMC) initialement bas prennent plus de poids : leur organisme retrouve progressivement un équilibre métabolique et nutritionnel en réponse à la suppression virale.\vspace{0.3cm} 

D'autres facteurs individuels sont également impliqués, comme l’âge : \textcite{erlandson2021weight} rapportent que les patients de moins de 35 ans, en particulier ceux à faible IMC, présentent une prise pondérale plus marquée. Le mode de vie est aussi important : l’étude de \textcite{kim2022weight} en Corée du Sud montre que les patients physiquement inactifs ou très immunodéprimés (CD4 < 100 cellules/mm³) présentent une prise de poids plus marquée.\\ 




\begin{itemize}
	\item[\ding{118}]	\textbf{ \large Facteurs contextuels et géographiques}
\end{itemize} 




Outre ces facteurs qui sont individuelles, le contexte dans lequel évoluent les patients influence fortement l’évolution pondérale sous TAR. Les conditions de vie, les pratiques de soins et l’état clinique initial jouent ici un rôle majeur. \textcite{huis2015determinants}, dans une étude portant sur plus de 200 000 patients suivis dans des contextes à faibles ressources (Afrique subsaharienne et Asie-Pacifique), rapportent que les individus en mauvais état général au début du traitement, avec une faible hémoglobine, un stade clinique avancé de l’infection ou une malnutrition, ont tendance à prendre du poids rapidement, surtout au cours de la première année. Ce phénomène est souvent interprété comme une récupération nutritionnelle plus qu’un effet indésirable du traitement.\vspace{0.3cm} 

De même, l’étude d’\textcite{ando2021} souligne que les patients asiatiques naïfs de traitement prennent du poids de façon significative, bien que les gains soient généralement moindres que ceux observés chez les patients d’autres origines ethniques. Ce constat suggère que la prise de poids dépend aussi du niveau de progression de la maladie au moment de l’initiation du traitement.\vspace{0.3cm} 




Enfin, les disparités régionales montrent que la géographie peut influencer la prise de poids chez les patients sous traitement antirétroviral. L’étude de \textcite{huis2015determinants} montre par exemple que les patients vivant en Afrique de l’Ouest et en Afrique centrale prennent significativement plus de poids que ceux d’Afrique australe ou d’Asie-Pacifique, malgré des schémas thérapeutiques comparables. Ces différences pourraient être liées à des habitudes alimentaires spécifiques, à un niveau de malnutrition initial différent ou à une accessibilité variable aux soins. \textcite{esber2022weight} observent également ces différences significatives entre le Nigeria, la Tanzanie et l'Ouganda. Les auteurs montrent une prise de poids plus importante chez les femmes au Nigeria et en Tanzanie par rapport aux femmes ougandaises après la transition vers un schéma à base de DTG.\\


En conclusion, la littérature montre que la prise de poids sous traitement antirétroviral est un phénomène multifactoriel. Les inhibiteurs de l’intégrase, et en particulier le dolutégravir, ainsi que inhibiteurs nucléosidiques de la transcriptase inverse (TAF), apparaissent comme les principaux facteurs pharmacologiques associés à une augmentation pondérale significative. Cependant, cette prise de poids n’est pas uniquement déterminée par le type de traitement : des facteurs individuels (sexe, origine ethnique, IMC initial, charge virale, le taux de CD4), contextuels (état clinique initial, accès aux soins, accompagnement nutritionnel) et géographiques (région, habitudes alimentaires, infrastructures sanitaires) modulent fortement les trajectoires pondérales individuelles. Ces résultats soulignent la nécessité d’une approche personnalisée dans la surveillance et la gestion métabolique des patients vivant avec le VIH, prenant en compte à la fois les caractéristiques du traitement et les spécificités individuelles et environnementales.








