%
% File: chap01.tex
% Author: Victor F. Brena-Medina
% Description: Introduction chapter where the biology goes.
%
\let\textcircled=\pgftextcircled
\chapter{PRÉSENTATION DES DONNÉES ET MÉTHODOLOGIE}
\label{chap:intro}
 
 

Dans ce chapitre, nous décrivons la source des données mobilisées dans le cadre de notre étude ainsi que la méthodologie employée pour déterminer les facteurs associés à la prise de poids dans l'essai MODERATO

\section{Présentation des données}

\vspace{0.4cm}

\subsection{Type de l’étude}


L’essai ANRS MIE MODERATO\footnote{\href{https://vih.org/vih-et-sante-sexuelle/20250722/premier-essai-dallegement-en-bitherapie-en-afrique-essai-anrs-moderato/}{Premier essai d’allègement en bithérapie en Afrique}} est un essai clinique randomisé, ouvert\footnote{Un essai ouvert (ou open-label trial) est un type d’essai clinique où toutes les personnes impliquées savent quel traitement est administré.} et multicentrique, mené en Afrique de l’Ouest et Centrale (Burkina Faso, Cameroun et Côte d’Ivoire). Il a été conçu pour évaluer la non-infériorité d’une bithérapie de maintenance (DTG/3TC ou atazanavir boosté [ATV/r] + 3TC) par rapport à une trithérapie standard (3DR~: ténofovir/3TC/éfavirenz, TLE, ou ténofovir/3TC/dolutégravir, TLD) chez des adultes vivant avec le VIH-1.
 \vspace{0.2cm}

Les patients étaient sous traitement antirétroviral (TLE ou TLD) depuis au moins deux ans, sans antécédent d’échec virologique ni hépatite B, et avec un taux de CD4+ supérieur à 200/mm³ à l’inclusion. L’objectif principal était le maintien d’une charge virale indétectable à S96 (<50 copies/ml).\vspace{0.2cm}

À la semaine 96, la bithérapie s’est révélée virologiquement non inférieure à la trithérapie, avec des taux de succès élevés et une bonne tolérance clinique et biologique. Ces résultats soutiennent la récente recommandation de l’OMS, depuis le 20 juillet 2025, d’adopter la bithérapie DTG/3TC comme option de maintenance dans les pays à ressources limitées.






\subsection{Population d’étude }



La population cible de cet essai comprenait 480 patients répondant aux critères d’inclusion présentés dans le Tableau~\ref{tab:critere}. Ces participants ont été randomisés en trois groupes égaux selon un schéma d’allocation 1:1:1, avec 160 patients par bras de traitement. Le premier groupe a reçu la bithérapie DTG + 3TC, le deuxième groupe a suivi la bithérapie ATV/r + 3TC, tandis que le troisième groupe a été traité par la trithérapie standard, comprenant soit TDF + 3TC + EFV, soit DTG + 3TC + TDF.

\begin{table}[H]
	\centering
	\caption{Critères  d’inclusion dans l’essai MODERATO}
	\renewcommand{\arraystretch}{1.3} % espace vertical
	\begin{tabular}{|c|p{13.2cm}|}
		\hline
		\textbf{N\textsuperscript{o}}
		 & \textbf{Critères d'inclusion} \\
		\hline
		1 & Infection documentée par le VIH-1. \\
		\hline
		2 & Âge supérieur ou égal à la majorité légale dans le pays participant. \\
		\hline
		3 & Numération des CD4 supérieure à 200 cellules/mm³ à la pré-inclusion. \\
		\hline
		4 & Traitement antirétroviral de première ligne depuis au moins deux ans, selon l’un des régimes suivants : 
		\begin{itemize}
			\item \textit{TDF+XTC+EFV} (sans antécédent d’échec virologique) ;
			\item \textit{TDF+XTC+EFV} suivi d’un switch vers \textit{DTG+XTC+TDF} ;
			\item Traitement initial par \textit{DTG+XTC+TDF}.
		\end{itemize} \\ \hline
		5 & Au moins deux charges virales consécutives inférieures au seuil de détection, dont celle de pré-inclusion. \\
		\hline
		6 & Pas d'antécédent d’échec virologique (CV $\geq$ seuil de détection ; tolérance de deux blips transitoires entre 50 et 200 copies/mL). \\
		\hline
		7 & Consentement éclairé signé. \\
		\hline
		8 & Pour les femmes en âge de procréer : engagement à utiliser une méthode contraceptive efficace pendant toute la durée de l’étude. \\
		\hline
	\end{tabular} 
	\label{tab:critere}
	\vspace{0.2cm}
	
	{\small \textbf{\textit{\underline{Source}}} : Protocole MODERATO, version 5.0}
\end{table}








\subsection{Schéma de suivi}


L’étude s’est déroulée du 21 septembre 2020 au 5 février 2025. Le schéma de suivi de l’essai ANRS~12372~MODERATO (Figure \ref{fig:suivi1}) a été conçu pour évaluer, sur 96 semaines (clôture prévue à la semaine 100), l’efficacité, la tolérance, l’observance et l’impact global des différentes stratégies antirétrovirales. Chaque participant a bénéficié d’un suivi structuré, comprenant des consultations cliniques, des examens biologiques, des questionnaires auto-administrés et la collecte d’échantillons biologiques.
\vspace{0.2cm}



\noindent Le calendrier de suivi comporte douze visites programmées :

\begin{itemize}[label=\textbullet] % classique
	\item Une visite de pré-inclusion (S-8) pour vérifier les critères d’éligibilité ;
	\item Une visite d’inclusion et de randomisation (J0) ;
	\item Dix visites de suivi aux semaines 4, 8, 12, 24, 36, 48, 72, 84, 96 et 100.
\end{itemize}
\newpage

\begin{figure}[H]
	\centering
	\caption{Schéma de suivi de l’essai MODERATO}
	\includegraphics[width=1\textwidth]{arbre.png}\\
	\label{fig:suivi1}
{\small \textbf{\underline{\textit{Source}}} : Protocole MODERATO, version 5.0}
\end{figure}


\section{Présentation des principales variables d’intérêt de l’étude} 


Pour mener à bien notre étude, les principales variables de cette analyse sont consignées dans le tableau~\ref{tab:variables}. Elles sont considérées comme pertinentes dans le cadre de cette analyse grâce au parcours de la littérature. Il s'agit de variables qui capturent à la fois les caractéristiques démographiques, cliniques, thérapeutiques et l'évolution temporelle des patients : \vspace{0.3cm}

--- \textbf{Pays} : Variable catégorielle indiquant le centre de suivi du patient. Elle permet de tenir compte des potentielles disparités géographiques entre la Côte d'Ivoire, le Burkina Faso et le Cameroun. \vspace{0.2cm}

--- \textbf{Sexe} : Le sexe est un facteur souvent rapporté comme associé à des différences dans la prise de poids sous traitement antirétroviral. \vspace{0.3cm}

--- \textbf{Âge} :  L'âge peut influencer le métabolisme de base et est un facteur confondant important à contrôler. \vspace{0.2cm}

--- \textbf{Durée du traitement ARV }:  Elle représente la durée écoulées depuis l'initiation du premier traitement antirétroviral jusqu'à l'inclusion dans l'essai. Cette variable permet d'évaluer l'impact de l'ancienneté du traitement. \vspace{0.2cm}

--- \textbf{Charge virale à l'inclusion} : Il s'agit de la dernière mesure de la charge virale avant la randomisation. Un contrôle virologique strict était un critère d'inclusion, cette variable est donc attendue comme très faible.\vspace{0.2cm}

--- \textbf{Date des visites} : Variable de type date qui sert de base au calcul du temps écoulé depuis l'inclusion pour les analyses longitudinales et de survie. \vspace{0.2cm}

--- \textbf{Poids à chaque visite} :  Il s'agit de la variable réponse principale de l'étude, utilisée pour modéliser l'évolution pondérale au cours du temps.\vspace{0.2cm}

--- \textbf{Régime ARV} : Le bras expérimental DTG + 3TC  ou le bras comparateur ATV/r + 3TC. C'est la variable explicative centrale pour tester la différence de prise de poids entre les deux schémas thérapeutiques. \vspace{0.2cm}

--- \textbf{CD4 à l'inclusion} : Elle reflète l'état immunitaire du patient au moment de commencer la bithérapie dans l'essai. \vspace{0.2cm}

--- \textbf{Nadir des CD4} : Il s'agit du taux le plus bas de lymphocytes CD4 jamais enregistré dans le dossier médical du patient. 

\begin{table}[H]
	\centering
		\caption{Présentation des variables de l’étude}
	\resizebox{\linewidth}{!}{
	\begin{tabular}{l|c|c}
		\hline
		\rowcolor[HTML]{C0C0C0} 
		\multicolumn{1}{|c|}{\cellcolor[HTML]{C0C0C0}\textbf{Variable}} & \multicolumn{1}{|c|}{\cellcolor[HTML]{C0C0C0}\textbf{Nature}}                                                 & \multicolumn{1}{c|}{\textbf{Mesure}} \\ \hline
		 
		                              &                               & 1- Côte d'Ivoire ;                                  \\ 
		
		                              &                                & 2-Burkina Faso;                                     \\ 
		 
		\multirow{-3}{*}{Pays}         & \multirow{-3}{*}{Catégorielle} & 3-Cameroun                                          \\ \hline
		
		\rowcolor[HTML]{EFEFEF} Sexe                                                   & Dichotomique                                           & 1-Masculin ; 2-Féminin                              \\ \hline
		
		 Âge                                                    & Discrète                                               & Années      \\ \hline
		
		
		  \rowcolor[HTML]{EFEFEF} Durée du traitement  ARV                  & Continue                                               &  Années                                                     \\ \hline
		
		
		
		
		 Charge virale  à l'inclusion                                & Discrète                                               &  Copies/ml                              \\ \hline
		
		
	
		\rowcolor[HTML]{EFEFEF}  Date des visites                             & Date                                               & Année-Mois-Jour                                    \\ \hline
		
		
		
		
		
		 Poids à chaque visite              & Continue                                               & kg                                   \\ \hline
		  \rowcolor[HTML]{EFEFEF}     Taille à chaque visite              & Continue                                               & Mètre                                   \\ \hline
	
		 
		                       &                                & 1- bras (DTG + 3TC) ;                             \\ 
		
		 \multirow{-2}{*}{Régime ARV}   & \multirow{-2}{*}{Dichotomique} & 2- bras (ATV/r + 3TC)                                              \\ \hline
		
		
		 
		 
		 
		 
		  \rowcolor[HTML]{EFEFEF}CD4 à l'inclusion          & Discrète                                               & Cellules/mm$^3$                                      \\ \hline
		 
		    Nadir CD4                                              & Discrète                                               &        Cellules/mm$^3 $           \\ \hline
		
	

	\end{tabular}} \vspace{0.1cm}
	\label{tab:variables} 

{\small \textbf{\underline{\textit{Source}}} : Données de l'essai MODERATO}
\end{table}

\section{Méthodologie de l’étude}

La présente étude repose sur une approche quantitative combinant différentes méthodes statistiques adaptées aux données cliniques. \vspace{0.2cm}

\begin{itemize}[label=\textbullet]
	\item \textbf{Analyse descriptive}
\end{itemize}


Une analyse descriptive a d’abord été réalisée pour caractériser les patients inclus, comprenant une approche univariée et bivariée. Dans l'analyse univariée, les variables quantitatives ont été résumées par la médiane, les premier et troisième quartiles (Q1-Q3) et les valeurs minimale et maximale [Min-Max], tandis que les variables qualitatives ont été présentées en effectifs et pourcentages. L'analyse bivariée a comparé ces caractéristiques selon deux axes : d'une part entre les deux bras thérapeutiques (DTG+3TC vs ATV/r+3TC) pour vérifier l'équilibre initial entre les groupes, et d'autre part selon le sexe des participants. Pour cette comparaison selon le sexe, les variables continues (âge, poids, IMC, CD4, durée du TARV) ont été comparées à l'aide du test de Wilcoxon-Mann-Whitney et sont présentées sous forme de médianes, tandis que les variables catégorielles (gain pondéral $\geq$5\%, bras thérapeutique) ont été comparées par le test du Chi-carré et présentées en effectifs et pourcentages. Parallèlement, l'évolution du gain moyen de poids depuis l'inclusion (J0) a été calculée pour chaque bras thérapeutique à chaque visite (S4 à S96). À la semaine 96, un test t a été appliqué séparément dans chaque bras pour vérifier si la variation moyenne différait significativement de zéro, et la comparaison inter-bras a été réalisée à l'aide d'un test t de Student. Ces analyses descriptives visaient à identifier des disparités potentielles qui pourraient influencer la prise de poids et devoir être prises en compte dans les analyses multivariées ultérieures.\\

\begin{itemize}[label=\textbullet]
	\item \textbf{Analyse longitudinale via modèle linéaire mixte}
\end{itemize}

Pour étudier l’évolution longitudinale du poids et tenir compte des variations individuelles, un modèle linéaire mixte a été utilisé. Ce modèle permet d’inclure à la fois les effets fixes des variables explicatives (traitement, caractéristiques individuelles) et les effets aléatoires liés aux différences non observées entre patients, afin d’évaluer si le traitement par DTG + 3TC entraîne une prise de poids significativement plus importante que le traitement par ATV/r + 3TC. \\

\begin{itemize}[label=\textbullet]
	\item \textbf{Analyse de survie : prise de poids d'au moins 5\%}
\end{itemize}

Enfin, pour identifier les facteurs associés à un gain de poids d’au moins 5\%, une analyse de survie a été conduite. Dans une première étape, une analyse de Kaplan-Meier a été réalisée pour estimer la probabilité de rester en dessous du seuil de prise de poids, selon les différents bras de bithérapie. Les courbes de survie ont été comparées visuellement à l’aide du test du log-rank. Dans une seconde étape, un modèle de régression de Cox à risques proportionnels a été estimé pour identifier les facteurs associés à la survenue d’une prise de poids ($\geq 5\%$). Ce modèle semi-paramétrique quantifie l’effet des covariables sur le risque instantané de prise de poids.



\subsection{Modèle linéaire à effets mixtes}




Le modèle linéaire mixte introduit par \textcite{laird1982random} est la méthode la plus utilisée pour étudier l’évolution d’un marqueur au cours du temps. Il consiste à prendre en compte la corrélation entre les mesures répétées des sujets dans l’estimation des paramètres de population, c’est à dire de l’évolution moyenne au cours du temps et de l’association avec des variables explicatives.  \vspace{0.2cm} 


Dans notre population de \( N = 320 \) patients, considérons le vecteur de réponses \( Y_i = (Y_{i1}, ..., Y_{in_i}) \) correspondant aux \( n_i = 10 \) mesures répétées du poids pour le patient \( i \)  $(J0, S4, S08, S12, S24, S34, S48, S72, S84, S96)$. L’objectif est de modéliser simultanément l’évolution moyenne du poids selon le traitement et les trajectoires individuelles. Un effet aléatoire de pente, en plus de l’effet aléatoire d’interception, a été spécifié pour chaque patient, reflétant à la fois le niveau initial de poids (interception aléatoire) et la vitesse d’évolution individuelle au cours du temps (pente aléatoire). Cette spécification permet de mieux représenter l’hétérogénéité interindividuelle des trajectoires de poids.\\


\begin{itemize}
	\item[\ding{118}]	\textbf{Spécification du modèle}
\end{itemize}

La formulation générale du modèle linéaire mixte standard s’écrit de la manière suivante :

\[
Y_{ij} = X^{'}_{ij}\beta +  b_{0i} + b_{1i}t_{ij}  + \epsilon_{ij} \hspace{0.5cm} avec \hspace{0.2cm} b_i = ( b_{0i} , b_{1i})^{'} \hspace{0.2cm} et\hspace{0.2cm} t_{ij} = temps \tag{2.1}
\]

\[
Y_{ij} = X^{'}_{ij}\beta + Z^{'}_{i} b_i + \epsilon_{ij} \hspace{0.5cm} avec \hspace{0.3cm} \epsilon_{ij} \sim \mathcal{N}(0, \sigma^2) \hspace{0.3cm} et \hspace{0.3cm} b_i \sim \mathcal{N}(0,B) \tag{2.2}
\]

où \( X_{ij} \) est le vecteur de variables explicatives de dimension $p$ pour la mesure $j$ du patient $i$, \( \beta \) est le p-vecteur des effets fixes ; \( Z_{i} = (1 \hspace{0.3cm}t_{ij}) \) est un sous-vecteur de \( X_{ij} \) de dimension \( 2 \). Les \( b_i \) sont des 2-valeurs d'effets aléatoire distribués identiquement et indépendamment suivant une loi normale d’espérance \( 0 \) et de matrice de variance-covariance \( B \).\vspace{0.3cm} 

Nous considérons dans cette étude la formulation la plus générale admettant que les erreurs résiduelles sont corrélées en plus des effets aléatoire. Conditionnellement aux effets aléatoire, le modèle s'écrit sous forme vectorielle :   
\[
Y_{i} = X^{'}_{i}\beta +   Z^{'}_{i} b_i + \epsilon_{i} \hspace{0.5cm} avec \hspace{0.2cm} \epsilon_{i} \sim \mathcal{N}(0, \Sigma_i) \tag{2.3}
\]

Et la formulation marginale est : 

\[
Y_{i} = X^{'}_{i}\beta +  \varepsilon_{i} \hspace{0.5cm} avec \hspace{0.2cm} \varepsilon_{i} \sim \mathcal{N}(0, V_i = Z_i B Z^{'}_i + \Sigma_i ) \tag{2.4}
\]

L'erreur $\varepsilon_{ij} $ peut être la somme de termes, un processus gaussien autorégressif capturant la corrélation résiduelle entre les mesures successives de Y et une erreur de mesure indépendante. Dans ce modèle, $X_i\beta$ représente le profil moyen pour la population des patients ayant les mêmes valeurs des variables X, les effets aléatoires représentent la tendance individuelle à long terme, l'erreur autorégressif représente les variations individuelles à court terme.\\

\newpage
\begin{itemize}
	\item[\ding{118}] \textbf{Estimation des paramètres}
\end{itemize}



Les paramètres du modèle linéaire mixte sont estimés par maximisation de la vraisemblance. La vraisemblance d'un modèle linéaire mixte (ou plus générallement d'un modèle multivarié normal) peut être obtenue en utilisant la formulation marginale du modèle $Y_{i} = X^{'}_{i}\beta +  \varepsilon_{i}$ avec $\varepsilon_{i} \sim \mathcal{N}(0, V_i)$. On note $\phi$ le vecteur des paramètres de covariance intervenant dans $V_i$. La log-vraisemblance s'écrit : 
\begin{align*}
	L(\beta,\phi) &= \sum_{i=1}^{N} \log \big( f_{Y_i}(Y_i) \big) \\
	&= \frac{1}{2} \sum_{i=1}^{N} \left[ 
	n_i \log(2\pi) + \log|V_i| + (Y_i - X_i \beta)^{'} V_i^{-1} (Y_i - X_i \beta) 
	\right] \tag{2.5}
\end{align*}



L'équation de score pour les paramètres $\beta$ est : 


\begin{align*}
	\frac{\partial L(\beta,\phi)}{\partial \beta} &= \sum_{i=1}^{N} X_i^\prime V_i^{-1} (Y_i - X_i \beta) = 0 \tag{2.6}
	\label{eq:score_beta}
\end{align*}


Lorsque les paramètres $\phi$ sont connus, $\beta$ est estimé par :

\begin{align*}
	\hat{\beta} &= \left( \sum_{i=1}^{N} X_i^\prime V_i^{-1} X_i \right)^{-1} 
	\left( \sum_{i=1}^{N} X_i^\prime V_i^{-1} Y_i \right) \tag{2.7}
	\label{eq:beta_hat}
\end{align*}





\vspace{0.5cm}

\begin{itemize}
	\item[\ding{118}]	\textbf{Adéquation du modèle}
\end{itemize}




Le modèle linéaire mixte présenté suppose une relation linéaire entre la variable dépendante et les covariables, aussi bien pour les effets fixes que pour les effets aléatoires. Il prend en compte une éventuelle dépendance entre les effets aléatoires et les erreurs résiduelles, tout en considérant que ces composantes suivent une distribution normale. Le modèle présume également une homoscédasticité conditionnelle des erreurs. Enfin, une spécification adéquate de la structure de covariance est essentielle pour bien capturer la corrélation intra-individuelle et garantir la validité des inférences.





\subsection{Analyse de survie non paramétrique : méthode de Kaplan-Meier}




L’analyse de survie non paramétrique, en particulier la méthode  \textcite{kaplan1958nonparametric}, constitue un outil fondamental pour étudier la distribution du temps jusqu’à la survenue d’un événement d’intérêt. Dans le cadre de cette étude, l’événement est défini comme une prise de poids cliniquement significative, c’est-à-dire une augmentation de poids $\geq 5\%$ par rapport au poids initial. La méthode de Kaplan-Meier, également appelée estimateur produit-limite, permet d’estimer la fonction de survie $S(t)$, soit la probabilité de ne pas encore avoir connu l’événement à un instant $t$, en tenant compte de la présence possible de données censurées à droite (par exemple, lorsque le suivi du patient est interrompu avant l’occurrence de la prise de poids). \vspace{0.3cm} 

Le principe repose sur le calcul de probabilités conditionnelles de survie entre des instants successifs $t_1 < t_2 < \dots < t_n$ où des événements sont observés. La probabilité de survie au temps $t_n$ peut s’écrire :

\[
S(t_n) = \prod_{i=1}^{n} P(T > t_i \mid T > t_{i-1})(T > t_0)
\]

Si on souhaite la survie à n’importe quel temps \( t \) tel que \( t < t_n \), il suffit d’arrêter le produit précédent au dernier individu suivi juste avant \( t \);

\[
S(t) = \prod_{i=1,i<t}^{n} P(T > t_i|T > t_{i-1}).
\]

Chaque terme est estimé empiriquement par :

\[
P(T > t_i \mid T > t_{i-1}) = 1 - \frac{d_i}{n_i}
\]

\noindent Où $n_i$ est le nombre de patients encore à risque (c’est-à-dire n’ayant pas encore connu l’événement ni été censurés) juste avant le temps $t_i$ et  $d_i$ est le nombre de patients ayant connu l’événement à $t_i$
%\begin{itemize}
%	\item $n_i$ est le nombre de patients encore à risque (c’est-à-dire n’ayant pas encore connu l’événement ni été censurés) juste avant le temps $t_i$,
%	\item $d_i$ est le nombre de patients ayant connu l’événement (prise de poids) à $t_i$.
%\end{itemize}  

\vspace{0.3cm} 


L’estimateur de Kaplan-Meier de la probabilité de survivre au moins jusqu'en t est donc : 

\[
\hat{S}(t) = \prod_{i=1,i<t}^{n} \left(1 - \frac{d_i}{n_i}\right).
\]

La fonction de survie ne varie que lorsqu’un événement observé survient (et non lors d’une censure). Cette méthode permet ainsi d’obtenir une estimation non paramétrique robuste de la probabilité de rester en dessous du seuil de prise de poids significative au cours du temps.


\subsection{Modèle de régression de Cox à risques proportionnels}

Pour aller au-delà de la simple comparaison descriptive des courbes de survie, le modèle de \textcite{cox1972regression} à risques proportionnels constitue une approche semi-paramétrique puissante permettant d’évaluer l’effet de plusieurs facteurs explicatifs sur le risque instantané de survenue d’un événement (un de gain de 5\% du poids initial). Adapté aux études longitudinale comme les études de cohorte et les essais thérapeutiques, le modèle de Cox est devenu le modèle de référence pour l'analyse statistique des enquête de cohorte en épidémiologie.\\

\newpage

\begin{itemize}
	\item[\ding{118}]	\textbf{Spécification du modèle}
\end{itemize}


La relation entre la fonction de risque associée à la survenue d’un évènement et le vecteur des \( p \) variables explicatives \( Z = (Z_1, Z_2, \ldots, Z_p)^T \) est la suivante :

\[\alpha(t, Z, \beta) = \alpha_0(t) r(\beta, Z)\]

où \( \beta = (\beta_1, \ldots, \beta_p)^T \) est le vecteur des coefficients de régression et \( \alpha_0(t) \) est appelée la fonction de risque de base. La fonction \( r(\beta, Z) \) dépend des caractéristiques \( Z \) du patient. En général, on prend \( r(\beta, Z) = \exp(\beta_1 Z_1 + \cdots + \beta_p Z_p) \) de façon à obtenir une fonction de risque positive sans contrainte sur les coefficients \( \beta \) quelles que soient les valeurs de \( Z \).\\

 Le modèle s’écrit alors :

\[\alpha(t, Z, \beta) = \alpha_0(t) \exp(\beta^T Z)\]




Dans ce cas, \( \alpha_0(t) \) est la fonction de risque des patients pour lesquels toutes les variables explicatives \( Z_j \) (\( j = 1, \ldots, p \)) sont nulles (si cela a un sens). Par construction du modèle, le rapport des fonctions de risque de deux patients \( i \) et \( j \) avec des vecteurs de variables explicatives \( Z_i \) et \( Z_j \) :

\[\frac{\alpha_i(t, Z_i, \beta)}{\alpha_j(t, Z_j, \beta)} = \frac{\alpha_0(t) \exp(\beta^T Z_i)}{\alpha_0(t) \exp(\beta^T Z_j)} = \frac{\exp(\beta^T Z_i)}{\exp(\beta^T Z_j)}\]

est constant au cours du temps, ce qui entraîne que les fonctions de risque sont proportionnelles d’où le nom du modèle. C’est une hypothèse qu’il faudra vérifier si on veut que les résultats du modèle soient valables.\\


\begin{itemize}
	\item[\ding{118}] \textbf{Estimation des paramètres}
\end{itemize}

\textcite{cox1972regression} a proposé d’estimer le vecteur \( \beta \) par maximisation d’une vraisemblance partielle. La vraisemblance partielle ne dépend pas de \( \alpha_0(t) \) ce qui permet d’estimer les \( \beta \) sans faire d’hypothèse sur \( \alpha_0(t) \). La vraisemblance partielle est obtenue en factorisant la vraisemblance et en n’en retenant qu’une partie qui n’implique pas \( \alpha_0(t) \).  En effet, soient \( t_1 < t_2 < \ldots < t_k \) les différents temps d’évènements observés et \( Z_{(1)}, \ldots, Z_{(k)} \) les variables explicatives des patients ayant subi l’évènement respectivement en \( t_1, \ldots, t_k \). La probabilité conditionnelle que le patients \( i \) subisse l’évènement en \( t_i \) sachant qu’il est à risque au temps \( t_i \) et qu’il n’y a qu’un seul évènement en \( t_i \) est égale à :

\[p_i = \frac{\alpha(t_i, Z_{(i)}, \beta)}{\sum_{t_i: \tilde{T}_i \geq t_i} \alpha(t_i, Z_i, \beta)}\]

En utilisant le modèle à risques proportionnels la probabilité conditionnelle devient :

\[p_i = \frac{\alpha_0(t_i) \exp(\beta^T Z_{(i)})}{\sum_{t_i: \tilde{T}_i \geq t_i} \alpha_0(t_i) \exp(\beta^T Z_i)} = \frac{\exp(\beta^T Z_{(i)})}{\sum_{t_i: \tilde{T}_i \geq t_i} \exp(\beta^T Z_i)}\]

Cette quantité ne dépend donc pas de la fonction de risque de base \( \alpha_0(t) \) qui est considérée ici comme un paramètre de nuisance. La vraisemblance partielle est alors obtenue comme le produit des probabilités conditionnelles calculées à chaque temps d’évènement :

\[\mathcal{L}(\beta, Z) = \prod_{i=1}^{k} p_i = \prod_{i=1}^{k} \frac{\exp(\beta^T Z_{(i)})}{\sum_{l: \tilde{T}_l \geq t_i} \exp(\beta^T Z_l)} \tag{4.7}\]



La vraisemblance partielle du modèle de Cox nécessite l’hypothèse de données continues, c’est-à-dire qu’il n’y a pas plusieurs évènements (ex aequo) à la même date. En pratique, comme dans notre étude, cette hypothèse n’est pas respectée en raison de la mesure du temps par intervalles lors de la collecte des données. Plusieurs corrections de la vraisemblance partielle ont été proposées. La méthode la plus utilisée est dénommée approximation de Breslow. On approche la vraisemblance partielle par :

\[\tilde{\mathcal{L}}(\beta, Z) = \prod_{i=1}^{k} \frac{\exp(\beta^T s_i)}{\left[ \sum_{l: \tilde{T}_l \geq t_i} \exp(\beta^T Z_l) \right]^{m_i}}\]

où \( s_i \) est le vecteur de la somme des vecteurs des variables explicatives des \( m_i \) patients ayant subi l’évènement au temps \( t_i \). \\


\textbf{ \underline{Interprétation des paramètres}} : Risques relatifs \vspace{0.2cm}


\begin{itemize}[label=\textbullet] % classique
	
	\item Si le modèle inclut une seule variable \( Z \)
	
	\begin{itemize}
		
		\item Si \( Z \) est dichotomique avec un codage 0/1, le risque relatif d’un sujet pour lequel \( Z = 1 \) par rapport à un patient pour lequel \( Z = 0 \) est obtenu par : \[
		\begin{aligned}
			RR(t) &= \frac{\alpha(t, Z = 1, \beta)}{\alpha(t, Z = 0, \beta)} \\
			&= \frac{\alpha_0(t) \exp(1 \times \beta)}{\alpha_0(t) \exp(0 \times \beta)} \\
			&= \frac{\alpha_0(t) \exp(\beta)}{\alpha_0(t)} \\
			&= \exp(\beta) \\
			&= RR
		\end{aligned}
		\]
		\end{itemize}
		Donc quel que soit \( t \), le risque instantané d’un patient pour lequel \( Z = 1 \) est égal au risque instantané d’un patient pour lequel \( Z = 0 \) multiplié par RR.
		
			\begin{itemize}
		\item   Si \( Z \) est une variable continue, \(\exp(\beta)\) correspond au risque relatif pour une augmentation de 1 unité de la variable. C’est donc le risque relatif d’un patient pour lequel \( Z = z + 1 \) par rapport à un patient pour lequel \( Z = z \).
		
		\end{itemize}
	
	\item Dans un modèle à p variables explicatives, le modèle s’écrit : 
	
\end{itemize}



\[\alpha(t, Z, \beta) = \alpha_0(t) \exp(\beta_1 Z_1 + \ldots + \beta_p Z_p)\]

et on a \(RR_k = \exp(\beta_k)\) qui est le risque relatif lié à la variable \(Z_k\) ajusté sur toutes les autres variables explicatives, c’est-à-dire le risque relatif d’un patient par rapport à un autre patient qui ne diffère que par la valeur de \(Z_k\). \\






\begin{itemize}
	\item[\ding{118}]	\textbf{Adéquation du modèle}
\end{itemize} 


Les deux principales hypothèses pour le modèle de Cox sont la proportionnalité des risques et la log-linéarité.  L’hypothèse de proportionnalité des risques signifie que les fonctions de risques pour les différentes modalités d’une variable sont proportionnelles et que leur rapport est indépendant du temps. Elle peut être vérifiée à l’aide de diagnostics, notamment les résidus de \textcite{schoenfeld1982partial}. Par ailleurs, le modèle  fait l’hypothèse de log-linéarité entre la fonction de risque et les variables explicatives quantitatives. Si \( Z \) est une variable quantitative, \(\exp(\hat{\beta})\) correspond au risque relatif pour une augmentation de 1 unité de la variable quelle que soit la valeur de la variable explicative ; c’est une hypothèse relativement forte qu’il convient de vérifier.






