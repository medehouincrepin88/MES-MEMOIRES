\pagestyle{plain}
\chapter*{CONCLUSION}\addcontentsline{toc}{chapter}{CONCLUSION}


En définitive, cette étude s’inscrit dans le cadre de l’essai clinique MODERATO, dont l’objectif était d’évaluer l’efficacité de stratégies thérapeutiques allégées chez les personnes vivant avec le VIH en Afrique de l’Ouest et centrale. Ainsi, notre travail s’est concentré sur les facteurs pouvant influencer le poids des patients dans cet essai ; plus précisément ceux traités par bithérapie : le dolutégravir + lamivudine (DTG+3TC) et l’atazanavir/ritonavir + lamivudine (ATV/r+3TC).\vspace{0.2cm}

L’analyse a été menée à partir de données longitudinales issues d’un échantillon de 320 patients, répartis de manière équitable entre les deux bras thérapeutiques. Le suivi s’est étendu sur 96 semaines, permettant une évaluation rigoureuse de l’évolution pondérale sous bithérapie. Afin de tirer pleinement parti de la richesse de ces données, plusieurs méthodes statistiques complémentaires ont été mobilisées. Le modèle linéaire mixte a permis d’étudier les trajectoires individuelles de poids au cours du temps et le modèle de régression de Cox multivarié a permis d’identifier les déterminants associés à une prise d'au moins 5\%. Ces différentes approches ont convergé vers des résultats cohérents et informatifs. \vspace{0.2cm}


L’analyse des résultats permet de dégager deux constats essentiels. Tout d’abord, les patients traités par la combinaison DTG+3TC ont montré une tendance à prendre davantage de poids que ceux sous ATV/r+3TC. Toutefois, cette différence, bien que présente, s’est révélée plus modérée que ce qui a été observé dans certaines études, suggérant une spécificité contextuelle des effets métaboliques dans notre contexte. Ensuite, plusieurs caractéristiques individuelles se sont avérées fortement liées à la prise de poids. Les femmes, les patients présentant un poids initial relativement faible et ainsi que les patients traités au Burkina, ont été susceptible de prise pondérale d'au moins 5\%. Ces résultats confirment l’influence de facteurs démographiques et cliniques sur les réponses au traitement antirétroviral.  \vspace{0.2cm}


Ces résultats revêtent une portée clinique importante pour la prise en charge des personnes vivant avec le VIH dans les pays d’Afrique subsaharienne. Ils plaident en faveur d’un suivi métabolique renforcé, en particulier chez les femmes, les patients présentant un poids initial relativement faible et surtout s'ils sont traités par dolutégravir ; cette dernière molécule étant le traitement de choix recommandé par l’OMS en tri comme en bi-thérapie. Des actions de prévention, incluant un accompagnement nutritionnel adapté et la promotion de l’activité physique, pourraient contribuer à limiter les effets métaboliques indésirables associés au traitement. \newpage

En somme, cette étude apporte des éléments de réflexion essentiels pour une optimisation thérapeutique contextualisée et individualisée. Elle met en lumière l’importance d’une approche personnalisée du traitement antirétroviral, articulant efficacité virologique, tolérance métabolique et qualité de vie. Ces résultats s’inscrivent dans une perspective de médecine de précision et de prévention, adaptée aux réalités cliniques et épidémiologiques des pays africains.
%
%
%
%
%\section*{Recommandations}
%
%À la lumière de ces résultats, plusieurs recommandations peuvent être formulées à destination des cliniciens, chercheurs et décideurs en santé publique : \\
%
%\textbf{Suivi métabolique renforcé des femmes et des patients âgés :} Ces populations présentent un risque accru de prise de poids significative sous bithérapie. Une surveillance régulière du poids, de la composition corporelle et des paramètres métaboliques est donc indispensable. \vspace{0.2cm}
%	
% \textbf{Intégration du poids initial dans la décision thérapeutique :} Les patients ayant un faible poids à l’inclusion devraient faire l’objet d’un suivi nutritionnel spécifique, afin de différencier un gain pondéral bénéfique d’une évolution vers le surpoids ou l’obésité. \vspace{0.2cm}
%	
%	
%\textbf{Renforcement de la recherche en contexte africain :} Les effets métaboliques des ARV, largement documentés en Occident, doivent être étudiés plus systématiquement en Afrique subsaharienne où les facteurs nutritionnels, génétiques et socioéconomiques peuvent modifier les trajectoires de poids.
%	

%
%Enfin, il serait pertinent que les futures recommandations cliniques intègrent la dimension métabolique comme critère de choix des schémas thérapeutiques, en particulier dans les stratégies d’allègement à visée long terme. La mise en œuvre de ces recommandations contribuera à améliorer la qualité de vie des patients, tout en consolidant les acquis thérapeutiques dans la lutte contre le VIH.