\pagestyle{plain}


\appendix
\renewcommand{\thefigure}{A\arabic{figure}}  % Pour les figures
\renewcommand{\thetable}{A\arabic{table}}    % Pour les tableaux
\setcounter{figure}{0}  % Remet à zéro
\setcounter{table}{0}
\chapter{ANNEXE }%\addcontentsline{toc}{chapter}{ANNEXE A}
\label{chap:Annexe}





\section*{Critère de non-inclusion}

\begin{table}[H]
	\centering
	\caption{Critères de non-inclusion dans l’essai MODERATO}
	\renewcommand{\arraystretch}{1.3} % espace vertical
	\begin{tabular}{|c|p{13.2cm}|}
		\hline
		\textbf{N\textsuperscript{o}} & \textbf{Critères de non-inclusion} \\
		\hline
		1 & Infection par le VIH-2 ou VIH-1+2. \\
		\hline
		2 & Nadir des CD4 < 100 cellules/mm\textsuperscript{3}. \\
		\hline
		3 & Hépatite chronique B (Ag HBs positif dans le bilan de pré-inclusion). \\
		\hline
		4 & Tuberculose pulmonaire active en cours de traitement. \\
		\hline
		5 & Infection opportuniste en cours d’évolution. \\
		\hline
		6 & Chimiothérapie ou immunothérapie en cours. \\
		\hline
		7 & Anomalie biologique de grade > 2 sur les paramètres usuels de la numération formule sanguine : taux d’hémoglobine, neutrophiles, plaquettes. \\
		\hline
		8 & ALAT $\geq$ 3 fois la limite supérieure de la normale, ou critères d’insuffisance hépatique sévère. \\
		\hline
		9 & Clairance de la créatinine < 50 mL/min (formule CKD-EPI). \\
		\hline
		10 & Allergie à un des médicaments de l’étude ou à leurs composants. \\
		\hline
		11 & Prise de traitements contre-indiqués avec les médicaments de l’étude. \\
		\hline
		12 & Grossesse en cours ou refus de contraception. \\
		\hline
		13 & Personne à risque de non-compliance. \\
		\hline
		14 & Symptômes ou valeurs biologiques évoquant un trouble systémique (rénal, hépatique, cardiovasculaire, pulmonaire), ou toute autre affection médicale susceptible d’interférer avec l’interprétation des résultats de l’essai ou de compromettre la santé des patients. \\
		\hline
	\end{tabular} \vspace{0.3cm}
	
	{\small \textbf{\underline{\textit{Source}}} : Protocole MODERATO, version 5.0}
\end{table}



\newpage



\section*{Analyse descriptive}




\begin{table}[H]
	\centering
			\setlength{\tabcolsep}{55pt} % réduit l'espacement horizontal
	\renewcommand{\arraystretch}{1.2}
	\caption{Pourcentage de patients ayant pris au moins 5\,\% de poids}
	\label{tab:poids_pourcentages}
	\begin{tabular}{cc}
		\hline
		\textbf{Semaine} & \hspace{2cm} \textbf{Prise de poids $\geq$ 5\,\% } \\
		\hline
		\rowcolor[HTML]{EDEDED} S4   & 7.19   \\
		S8   & 14.06  \\
		\rowcolor[HTML]{EDEDED} S12  & 18.13  \\
		S24  & 36.56 \\
		\rowcolor[HTML]{EDEDED} S36  & 40.31  \\
		S48  & 40.31   \\
		\rowcolor[HTML]{EDEDED} S72  & 43.44  \\
		S84  & 49.38   \\
		\rowcolor[HTML]{EDEDED} S96  & 45.94 \\
		\hline
	\end{tabular}
				{\small \textbf{\underline{\textit{Source}}} : Calculs de l'auteur }
\end{table}


Le tableau \ref{tab:poids_pourcentages} présente l'évolution du pourcentage de patients ayant pris au moins 5\,\% de leur poids initial entre la semaine 4 et la semaine 96. On observe une augmentation progressive de ce pourcentage au fil des semaines, passant de 7,2\,\% à S4 à près de 46\,\% à S96. Cette tendance indique que la majorité des participants a commencé à prendre du poids après la 12\textsuperscript{e} semaine, avec une hausse plus marquée entre S12 et S36. La légère baisse observée à S96 par rapport à S84 (45,94\,\% vs 49,38\,\%) pourrait refléter des variations individuelles ou des pertes de suivi.



\vspace{0.5cm}


\begin{table}[H]
	\centering
		\setlength{\tabcolsep}{40pt} % réduit l'espacement horizontal
	\renewcommand{\arraystretch}{1.4}
	\caption{Évolution de la prévalence du surpoids et de l'obésité entre J0 et S96}
	\label{tab:imc}
	\begin{tabular}{lcc}
		\toprule
		\textbf{Visite} & \textbf{Surpoids (\%)} & \textbf{Obésité (\%)} \\
		\midrule
	\rowcolor[HTML]{EDEDED} 	Jour 0  & 32.5  & 18.44 \\
		Semaine 96 & 32.91 & 26.52 \\
		\bottomrule
	\end{tabular}
				{\small \textbf{\underline{\textit{Source}}} : Calculs de l'auteur }
\end{table}





Le surpoids et l’obésité sont définis à partir de l’Indice de Masse Corporelle (IMC), qui se calcule en divisant le poids (en kg) par la taille au carré (en m\textsuperscript{2}). Selon les critères de l’Organisation mondiale de la santé (OMS), on parle de surpoids lorsque l’IMC est compris entre 25 et 29{,}9, et d’obésité lorsque l’IMC est supérieur ou égal à 30.  

Le tableau ci-dessous montre une augmentation notable de la proportion de patients en surpoids et obèses entre le début du traitement (J0) et la semaine 96. La prévalence du surpoids est passée de 32.5 \% à 32.91 \%, tandis que celle de l’obésité a augmenté de 18.44\,\% à 26.52\,\%. 



\section*{Traitement des valeurs manquantes}


\begin{figure}[H]
	\centering
	\caption{Résumé des valeurs manquantes}
	\includegraphics[width=1\textwidth]{manquant1.png}\\
	\label{fig:manquant1}
				{\small \textbf{\underline{\textit{Source}}} : Calculs de l'auteur }
\end{figure}

Pour les poids manquants, on a remplacé chaque valeur absente par la moyenne des poids mesurés à la visite précédente et à la visite suivante.

\begin{figure}[H]
	\centering
	\caption{Résumé des valeurs manquantes après imputation}
	\includegraphics[width=1\textwidth]{manquant2.png}\\
	\label{fig:manquant2}
				{\small \textbf{\underline{\textit{Source}}} : Calculs de l'auteur }
\end{figure}

Les poids manquants à la visite S96 ont été imputés à l’aide d’un modèle linéaire mixte, où le poids est la variable dépendante et le temps (en jours depuis la visite J0) est la variable explicative. Le modèle inclut un intercept et une pente aléatoires pour chaque individu, ce qui permet de prendre en compte les différences entre participants dans l’évolution de leur poids au fil du temps.

\begin{figure}[H]
	\centering
	\caption{Poids observé vs Poids prédit à la semaine 96}
	\includegraphics[width=1\textwidth]{manquant3.png}\\
	\label{fig:manquant3}
				{\small \textbf{\underline{\textit{Source}}} : Calculs de l'auteur }
\end{figure}

\begin{figure}[H]
	\centering
	\includegraphics[width=1\textwidth]{manquant4.png}\\
	\label{fig:manquant4}
				{\small \textbf{\underline{\textit{Source}}} : Calculs de l'auteur }
\end{figure}

\newpage

\section*{Résultat et Adéquation du modèle linéaire à effet mixte}


\begin{figure}[H]
	\centering
		\caption{Résultat du modèle linéaire à effet mixte }
	\includegraphics[width=1\textwidth]{mlm1.png}\\
	\label{fig:mlm}
				{\small \textbf{\underline{\textit{Source}}} : Calculs de l'auteur }
\end{figure}


\newpage


\begin{table}[H]
	\centering
	\caption{Comparaison des résultats des modèles mixtes (avant et après nettoyage des données)}
		\setlength{\tabcolsep}{10pt} % Espacement horizontal
	\renewcommand{\arraystretch}{1.1} % Espacement vertical
	\resizebox{\linewidth}{!}{
	\begin{tabular}{lccc}
		\hline
		\textbf{Paramètres} & \textbf{Modèle brut} & \textbf{Modèle nettoyé} & \textbf{Commentaire} \\
		\hline
	\rowcolor[HTML]{D9D9D9}	\multicolumn{4}{l}{\textit{\textbf{Informations générales}}} \\
		Nombre de sujets          & 320    & 320    & Identique \\
	\rowcolor[HTML]{F2F2F2}	Nombre d’observations     & 3200   & 3181   & 19 observations supprimées \\
		Log-vraisemblance         & -7915.73 & -7570.36 & Meilleur ajustement après nettoyage \\
	\rowcolor[HTML]{F2F2F2}	AIC                       & 15861.46 & 15170.71 & Amélioration du modèle \\
		BIC                       & 15917.98 & 15227.24 & Amélioration du modèle \\
		\hline
	\rowcolor[HTML]{D9D9D9}	\multicolumn{4}{l}{\textit{\textbf{Effets fixes}}} \\
		Intercept                 & 74.12*** & 74.00*** & Stable \\
		Jour (100 jours)          & 0.360*** & 0.362*** & Stable, effet positif \\
	\rowcolor[HTML]{F2F2F2}	Traitement DTG + 3TC      & -2.04 (ns) & -2.07 (ns) & Non significatif \\
		Sexe féminin              & -3.65 \textit{(p=0.063)} & -3.60 \textit{(p=0.067)} & Tendance négative \\
	\rowcolor[HTML]{F2F2F2}	Âge [50-60[               & 1.05 (ns) & 0.98 (ns) & Non significatif \\
		Âge [60-78]               & -1.02 (ns) & -0.94 (ns) & Non significatif \\
		\rowcolor[HTML]{F2F2F2} CD4   & 0.00617** \textit{(p=0.020)}& 0.00619** \textit{(p=0.020)} & Positif, significatif \\
		Durée TARV                & -0.522** & -0.524** & Négatif, significatif \\
		Jour × DTG+3TC            & 0.189*\textit{(p=0.027)}  & 0.193* \textit{(p=0.020) }  & Plus significatif après nettoyage \\
		\hline
	\rowcolor[HTML]{D9D9D9}	\multicolumn{4}{l}{\textit{\textbf{Effets aléatoires}}} \\
		Var(intercept)            & 196.80 & 192.23 & Très proche \\
		Cov(intercept, pente)     & 0.33   & 0.63   & Différence notable \\
		Var(pente jour\_100)      & 0.1206 & 0.0021 & Forte réduction de l’hétérogénéité \\
		\hline
	\rowcolor[HTML]{D9D9D9}	\multicolumn{4}{l}{\textit{\textbf{Structure des erreurs}}} \\
		Paramètre AR              & 0.325  & 0.185  & Corrélation temporelle plus faible \\
		Erreur résiduelle         & 0.929  & 0.705  & Forte amélioration \\
		\hline
	\end{tabular}}

{\small \textbf{\underline{\textit{Source}}} : Calculs de l'auteur }

	\label{tab:influents}
\end{table}




Après le retrait des valeurs influentes, le modèle mixte à mesures répétées s'ajuste mieux aux données, comme le montrent l'amélioration de la log-vraisemblance, la diminution des critères d'information (AIC et BIC) et la réduction de l'erreur résiduelle. Les effets fixes principaux restent stables et significatifs : le temps (\textit{jour\_100}) montre une progression moyenne du poids au cours du suivi, le taux de CD4 est positivement associé au poids, la durée du traitement antirétroviral (TARV) exerce un effet négatif, et l'interaction entre le temps et le traitement DTG+3TC reste significative, indiquant que l'effet du traitement se manifeste davantage au fil du temps. La variabilité interindividuelle sur la pente diminue fortement après nettoyage, révélant que certaines trajectoires atypiques influençaient de manière disproportionnée les estimations initiales. Enfin, la corrélation autorégressive entre mesures successives est réduite, ce qui reflète une meilleure capture de la dynamique longitudinale et rend le modèle final plus fiable, robuste et interprétable pour suivre l'évolution du poids des patients.


%Après avoir retiré les valeurs influentes, le modèle mixte à mesures répétées s'ajuste mieux aux données, avec des critères d'information (AIC et BIC) plus faibles et des résidus plus petits. Les effets significatifs, tels que le temps, le taux de CD4, la durée du TARV et l'interaction temps × traitement, restent stables. La variabilité entre les individus pour l'évolution du poids diminue, indiquant que certaines trajectoires atypiques avaient un effet exagéré. La dépendance entre mesures répétées est également réduite, ce qui rend le modèle plus fiable et facile à interpréter.

\newpage

\begin{figure}[H]
	\centering
	\caption{Normalité des résidus}
	\includegraphics[width=1\textwidth]{Residu.png}\\
	\label{fig:residusA}
				{\small \textbf{\underline{\textit{Source}}} : Calculs de l'auteur }
\end{figure}


Dans le cadre des modèles linéaires mixtes appliqués aux données longitudinales, la question de la normalité des résidus est souvent soulevée. Toutefois, il convient de préciser que les tests usuels de normalité, tels que ceux de Shapiro-Wilk (1965) ou de Jarque-Bera (1980), sont extrêmement sensibles à la taille de l’échantillon. Ainsi, dans le cas présent où l’on dispose de 320 individus suivis sur dix visites, soit 3 200 observations, ces tests conduisent presque inévitablement à un rejet de l’hypothèse nulle de normalité, même lorsque les écarts observés par rapport à la loi normale sont minimes et sans véritable conséquence pratique. Pour compléter cette analyse et évaluer l’hypothèse de normalité, une inspection graphique des résidus (histogramme, courbes de densité et Q-Q plots) a été réalisée(Figure~\ref{fig:residusA}). Ces représentations visuelles confirment que les résidus suivent globalement une distribution proche de la normal. Par ailleurs, comme l’indiquent \textcite{lumley2002importance, verbeke2000linear}, au-delà de 500 observations les intervalles de confiance et tests restent fiables, et de légères déviations à la normalité n’affectent pas la validité des modèles mixtes.

\vspace{2cm}


Dans la figure~\ref{fig:aleatoire}, si les lignes sont droites, cela signifie que la normalité est respectée, si elles sont courbe, cela signifie que la normalité n’est pas respectée. Dans notre cas, les variances des résidus sont plutôt distribuée en suivant une distribution normale.

\begin{figure}[H]
	\centering
	\caption{Normalité des effets aléatoires}
	\includegraphics[width=0.9\textwidth]{aleatoire.png}
	\label{fig:aleatoire}\\
				{\small \textbf{\underline{\textit{Source}}} : Calculs de l'auteur }
\end{figure}


\begin{figure}[h]
	\centering
	\caption{Performance du modèle mixte }
	\includegraphics[width=1\textwidth]{perf.png}\\
	\label{fig:perf}
				{\small \textbf{\underline{\textit{Source}}} : Calculs de l'auteur }
\end{figure}








%\begin{figure}[H]
%	\centering
%	\caption{Indépendance entre la variable aléatoire et le résidu}
%	\includegraphics[width=1\textwidth]{aleatoire1.png}\\
%	\label{fig:aleatoire1}
%\end{figure}
%
%
%Dans ce graphique ~\ref{fig:aleatoire1}, la diagonale représente les paramètres estimés. Les droites, plus ou moins verticales et horizontales, illustrent la manière dont un paramètre est estimé en fonction de la valeur conditionnelle d'un autre paramètre. En haut à gauche (2,1), le graphique montre comment la variance de l'intercept aléatoire \textit{\textbf{.sig01}} influence les résidus \textit{\textbf{.sigma}}. Sur la même ligne en (2,2), de façon similaire, il présente la relation entre la variance de la pente aléatoire \textit{\textbf{.sig02}} et les résidus. En (3,2), le graphique illustre la relation entre la covariance entre l'intercept et la pente aléatoires \textit{\textbf{.sig03}} et les résidus. Dans l'ensemble, ces résultats indiquent que les paramètres estimés sont indépendants : les résidus ne sont pas corrélés avec l'écart-type de l'intercept aléatoire, de la pente aléatoire, ni avec leur covariance.\\



%\begin{figure}[H]
%	\centering
%	\caption{Comparaison des modèles GLS et LME}
%	\includegraphics[width=1\textwidth]{interet_alea.png}\\
%	\label{fig:interet_alea}
%\end{figure}
%
%La comparaison des deux modèles indique que le modèle linéaire mixte (Modele.lme) présente une performance nettement meilleure que le modèle Modele.0. En effet, l’AIC passe de 26 115.09 à 15 803.33, et le BIC de 26 175.72 à 15 882.15, ce qui traduit une amélioration significative de l’ajustement. De plus, la log-vraisemblance augmente de -13 047.547 à -7 888.665. Le test du rapport de vraisemblance comparant les deux modèles fournit une statistique de 10 317.76 avec une p-value < 0.0001, confirmant que l’ajout des effets aléatoires dans le modèle Modele.lme améliore significativement sa capacité à expliquer les données. Par conséquent, le modèle Modele.lme est à privilégier pour mieux capturer la structure intra-individuelle des observations.\\




%
%
%\begin{figure}[H]
%	\centering
%	\caption{ Modèle linéaire mixte standard et Modèle avec variance hétérogène}
%	\includegraphics[width=1\textwidth]{hetero.png}\\
%	\label{fig:hetero}
%\end{figure}
%
%La comparaison des deux modèles, Modele.lme.hetero (avec hétéroscédasticité) et Modele.lme (homoscédasticité), indique qu’introduire une variance différente par bras ne permet pas une amélioration significative de l’ajustement du modèle. En effet, l’AIC passe de 15 803.33 à 15 804.15, et le BIC de 15 882.15 à 15 889.03, soit une variation négligeable. De plus, la log-vraisemblance augmente très légèrement de -7 888.665 à -7 888.076. Le test du rapport de vraisemblance (L.Ratio = 1.1793, p-value = 0.2775) n’est pas significatif, indiquant que l’hypothèse d’homoscédasticité ne peut pas être rejetée. Ainsi, le modèle Modele.lme reste le plus parcimonieux et le plus approprié pour décrire les données.

%\begin{figure}[H]
%	\centering
%	\caption{ Homoscédasticité : Résidus vs Valeurs ajustées}
%	\includegraphics[width=1\textwidth]{homo.png}\\
%	\label{fig:homo}
%\end{figure}

\begin{figure}[H]
	\centering
	\caption{Évolution individuelle de chaque patient selon le bras}
	\includegraphics[width=1\textwidth]{evolu.png}\\
	\label{fig:evolu}
					{\small \textbf{\underline{\textit{Source}}} : Calculs de l'auteur }
\end{figure}


%\newpage
%\begin{figure}[H]
%	\centering
%	\caption{Évolution individuelle de chaque patient}
%	\includegraphics[width=1\textwidth]{evolu2.png}\\
%	\label{fig:evolu2}
%\end{figure}
%
%Dans la figure~\ref{fig:evolu2}, on observe que la tendance dans l’évolution du poids varie d’un patient à l’autre au fil du temps. Cette variabilité justifie l’utilisation d’un modèle intégrant non seulement un intercept aléatoire, mais également une pente aléatoire pour chaque patient, afin de mieux capturer les différences individuelles dans les trajectoires de poids.
%
%\noindent Cela est confirmé par l’anova :
%
%
\begin{figure}[H]
	\centering
	\caption{Modèle sans pente aléatoire vs avec pente}
	\includegraphics[width=1\textwidth]{pente.png}\\
	\label{fig:pente}
					{\small \textbf{\underline{\textit{Source}}} : Calculs de l'auteur }
\end{figure}


La comparaison entre le modèle avec intercept aléatoire uniquement (Modele.lme.Sans) et celui avec intercept et pente aléatoires (Modele.lme) met en évidence une nette amélioration de l’ajustement lorsque la pente aléatoire est introduite. En effet, l’AIC diminue de 16 926.03 à 15 803.33, et le BIC de 16 992.72 à 15 882.15, ce qui reflète une meilleure qualité d’ajustement. De plus, la log-vraisemblance augmente significativement de -8 452.014 à -7 888.665. \vspace{0.3cm}

Le test du rapport de vraisemblance confirme cette amélioration avec une statistique de 1 126.698 et une p-value < 0.0001, indiquant que l’ajout d’une pente aléatoire améliore significativement le modèle. Ainsi, le modèle Modele.lme (avec intercept et pente aléatoires) est clairement à privilégier pour capturer la variabilité intra-individuelle des trajectoires de poids au cours du temps.





\section*{Résultat et Adéquation du modèle Cox}





\begin{figure}[H]
	\centering
	\caption{Modèle de Cox multivarié}
	\includegraphics[width=1\textwidth]{cox.png}\\
	\label{fig:suivi}
				{\small \textbf{\underline{\textit{Source}}} : Calculs de l'auteur }
\end{figure} 


\begin{figure}[H]
	\centering
	\caption{Test des risques proportionnels (Schoenfeld)}
	\includegraphics[width=0.5\textwidth]{sch.png}\\
	\label{fig:sch}
					{\small \textbf{\underline{\textit{Source}}} : Calculs de l'auteur }
\end{figure}


Le test de proportionnalité des risques basé sur les résidus de Schoenfeld montre que l’ensemble des covariables incluses dans le modèle respecte l’hypothèse de risques proportionnels, comme en témoigne la non-significativité des valeurs p associées (toutes supérieures à 0,05). Le test global (p = 0,52) confirme également l’absence de violation de cette hypothèse. 


\begin{figure}[H]
	\centering
	\caption{Hypothèse de log linéarité}
	\includegraphics[width=1\textwidth]{log.png}\\
	\label{fig:log_linearite}
					{\small \textbf{\underline{\textit{Source}}} : Calculs de l'auteur }
\end{figure}















